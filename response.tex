\documentclass{article}
\usepackage{amsmath,amssymb,amsfonts,amsthm,bm}
\usepackage{enumerate}
\RequirePackage[normalem]{ulem} 
\RequirePackage{color}\definecolor{RED}{rgb}{1,0,0}\definecolor{BLUE}{rgb}{0,0,1} 
\providecommand{\DIFaddtex}[1]{{\protect\color{blue}\uwave{#1}}} %DIF PREAMBLE
\providecommand{\DIFdeltex}[1]{{\protect\color{red}\sout{#1}}}                      %DIF PREAMBLE
\title{Response to Reviewer's Comments}
\author{}
\date{}
% \setlength{\parindent}{0em}
\setlength{\parskip}{1em}
\begin{document}

\maketitle

Authors appreciate careful reading of the manuscript by the reviewers, and are thankful for helpful suggestions for its improvement. Authors have modified the manuscript substantially in light of the reviewer's comments. All the modifications and changes made are highlighted in the Marked Revised Manuscript. Issues and concerns raised by the reviewers are discussed as follows:

\section*{Reviewer \#1}
\textit{The authors have fully addressed most of my comments. However, there are small issues and typos in the present manuscript, which need to be fixed before publishing on EABE. I recommend a minor revision.}

\textbf{Comment 1.} \textit{In Figure 8, the reference value of the Scodelis-Lo roof was wrong. This needs to be changed to -0.3006.}

\textbf{Response:} Authors appreciate the reviewer's identification of the incorrect reference value in Figure 8. This value has been corrected to $-0.3006$ on Page 26.

\textbf{Comment 2.} \textit{On Page 2, line 19, ”furhter” needs to be changed to ”further”.}

\textbf{Response:} Authors appreciate reviewer for pointing out this typo. The ``furhter" has been modified to ``further" on Page 2 (line 28), and the authors have double checked the manuscript to address other grammatical and type issues.

\textbf{Comment 3.} \textit{On Page 2, the authors discussed the nodally integrated meshfree methods. However, most of the node-based meshfree methods suffer from instability. Recent work [1] provides the idea about how to use nodal integration in meshfree methods to achieve stable and accurate results. It is suggested that authors discuss this together with the other methods that are originally cited to deliver a detailed literature review.}


\textbf{Response:} Authors thank the reviewer for highlighting the valuable work by Wang et al. It has to be acknowledged that the term ``node-based meshfree methods" used on Page 2 refers to meshfree approximations that construct the shape functions based on nodal information, rather than specifically node-based meshfree integration schemes. 
We agree that this work represents an important advancement in meshfree methods and have cited it as Ref. [18] on Page 2 (line 23). 
Additionally, we have clarified the terminology by replacing the original phrase with ''node-based MLS/RK approximations" on Page 2 (line 30).

\textbf{Comment 4.} \textit{On Page 2, line 45-46, ”In the absence of a meshfree node, accuracy enforcement might be poor.” this statement is confusing. Please clarify or rephrase it.}

\textbf{Reply:} Authors appreciate the reviewer's careful observation. We have revised this sentence and changed it to ``The accuracy maybe poor at locations away from the sample points" on Page 3 (line 53-54).

\textbf{Comment 5.} \textit{[2] also discussed the surface-type and volume-type Nitsche’s methods as well as the choice of the penalty parameters.}

\textbf{Response:} Thank you for highlighting the valuable work by Wang et al. We acknowledge that the Nitsche method, as employed in their work to ensure variational consistency between background and immersed domains in composite elasticity problems, utilizes artificial parameters. Since this aspect is not directly relevant to our proposed method, we believe it's more appropriate to discuss this as an extended application of the Nitsche method.
Therefore, we have added the following statement on page 3 (line 76-79):

``Additionally, the Nitsche's method has been successfully applied to maintain the variational consistency between different geometrical or material domains in problems with multiple patches [30] and composite materials [31]."
 
\textbf{Comment 6.} \textit{The authors well discussed the history of meshfree methods in dealing with the accuracy and boundary conditions. However, the literature review of shells is generally needed for the readers to understand the long history of developments. It is suggested that the authors create a small paragraph to discuss several milestones in the shell history. Thick shells by [3] (almost the first work); Well-known Hughes–Liu shell[4]; Belytschko—Lin—Tsay shell [5]; Geometry exact shell by Simo et al. [6]. [7] the first EFG paper for K–L shells;[8] first did the crack simulation based on the K–L shell theory;[9] presented a reformulation of K-L shell theory to accommodate arbitrary nonlinear materials; [10] first IGA K–L shell. Some of them are already in the original reference list.}

\textbf{Response:} Authors are thankful to the reviewer's suggestion to incorporate relevant discussion on conventional thin- and thick-shell formulation. We have added the following discussion on Page 2 (line 11-23): 

``Traditional finite element methods typically employ $C^0$ continuous shape functions, and it prefers hybrid and mixed shell models, like linear and nonlinear Mindlin model [2, 3] and the one inextensible director model [4]. 
Over the past thirty years, various novel formulations with high order smoothed shape functions have been applied to thin shell formulations. These include element-free Galerkin method [5], maximum-entropy meshfree method [6], Hermite reproducing kernel particle method [7], peridynamics [8], isogeometric analysis [9], and others.
For a more comprehensive review of advances and applications of high order formulations in various scientific and engineering fields, refer to [10, 12, 13, 14, 15, 16, 17, 18]."

\textbf{Comment 7.} \textit{Section 5.1, the thickness is $h = 0.1$ but the radius is $R = 1$, which the radio is slightly large for employing the K–L theory. In general, $R/h > 20$ is considered as thin shell [10]. Will it be possible to have the patch test done with a smaller thickness value?}

\textbf{Response:} Authors are thankful to the reviewer for the suggestion regarding the radius-thickness radio of thin shell. The shell thickness used in patch test has been changed to $h=0.05$, i.e. $\frac{R}{h} = 20$. Accordingly, the following elements have been updated in revised manuscript
\begin{itemize}
    \item The relevant statement regarding the shell thickness on Page 20 (line 366). 
    \item The patch test results in Table 1 and 2 on Pages 21 and 22.
    \item The moment contour plot in Figure 4 on Page 22. 
\end{itemize}

\section*{Reviewer \#2}
\textit{I have read the authors' revised manuscript and the associated responses to the review comments. The revised manuscript is well written and my review comments have been well addressed and implemented. I thereby recommend the acceptance of this manuscript.}

\textbf{Response:} Authors are thankful for the comments and suggestions, which definitely help the authors to improve this manuscript.

\section*{References}
\begin{enumerate}[{[1]}]
    \item J. Wang, M. Behzadinasab, W. Li, Y. Bazilevs, A stable formulation of correspondence-based peridynamics with a computational structure of a method using nodal integration, International journal for numerical methods in engineering (2024) e7465. 
    \item J. Wang, G. Zhou, M. Hillman, A. Madra, Y. Bazilevs, J. Du, K. Su, Con- sistent immersed volumetric nitsche methods for composite analysis, Com- puter Methods in Applied Mechanics and Engineering 385 (2021) 114042. 
    \item S. Ahmad, B. M. Irons, O. Zienkiewicz, Analysis of thick and thin shell structures by curved finite elements, International Journal for Numerical Methods in Engineering 2 (3) (1970) 419–451. 
    \item T. J. Hughes, W. K. Liu, Nonlinear finite element analysis of shells: Part I. three-dimensional shells, Computer Methods in Applied Mechanics and Engineering 26 (3) (1981) 331–362. 
    \item T. Belytschko, J. I. Lin, T. Chen-Shyh, Explicit algorithms for the non- linear dynamics of shells, Computer Methods in Applied Mechanics and Engineering 42 (2) (1984) 225–251.
    \item J. C. Simo, D. D. Fox, On a stress resultant geometrically exact shell model. Part I: Formulation and optimal parametrization, Computer Methods in Applied Mechanics and Engineering 72 (3) (1989) 267–304.
    \item P. Krysl, T. Belytschko, Analysis of thin shells by the element-free Galerkin method, International Journal of Solids and Structures 33 (20-22) (1996) 3057–3080.
    \item T. Rabczuk, P. Areias, T. Belytschko, A meshfree thin shell method for non-linear dynamic fracture, International Journal for Numerical Methods in Engineering 72 (5) (2007) 524–548.
    \item M. Behzadinasab, M. Alaydin, N. Trask, Y. Bazilevs, A general-purpose, inelastic, rotation-free Kirchhoff–Love shell formulation for peridynamics, Computer Methods in Applied Mechanics and Engineering 389 (2022) 114422.
    \item J. Kiendl, K.U. Bletzinger, J. Linhard, R. Wüchner, Isogeometric shell analysis with Kirchhoff–Love elements, Computer Methods in Applied Mechanics and Engineering 198 (49-52) (2009) 3902–3914.
\end{enumerate}
\end{document}
