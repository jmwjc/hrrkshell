%! TEX program = xelatexp
%DIF LATEXDIFF DIFFERENCE FILE
%DIF DEL main_v0.tex   Mon Mar 25 18:35:25 2024
%DIF ADD main.tex      Mon Mar 25 17:21:14 2024
%! TEX encoding = UTF-8 Unicode

\documentclass{elsarticle}
\usepackage{graphicx}
\usepackage{amsmath,amssymb,amsfonts,amsthm,bm}
\usepackage{hyperref}
\usepackage{booktabs,multirow}
\usepackage{lineno}
%DIF 10d10
%DIF < % \usepackage{cite}
%DIF -------
\linenumbers

\bibliographystyle{elsarticle-num}
%DIF < % \biboptions{longnamesfirst,angle,semicolon}
%DIF PREAMBLE EXTENSION ADDED BY LATEXDIFF
%DIF UNDERLINE PREAMBLE %DIF PREAMBLE
\RequirePackage[normalem]{ulem} %DIF PREAMBLE
\RequirePackage{color}\definecolor{RED}{rgb}{1,0,0}\definecolor{BLUE}{rgb}{0,0,1} %DIF PREAMBLE
\providecommand{\DIFaddtex}[1]{{\protect\color{blue}\uwave{#1}}} %DIF PREAMBLE
\providecommand{\DIFdeltex}[1]{{\protect\color{red}\sout{#1}}}                      %DIF PREAMBLE
%DIF SAFE PREAMBLE %DIF PREAMBLE
\providecommand{\DIFaddbegin}{} %DIF PREAMBLE
\providecommand{\DIFaddend}{} %DIF PREAMBLE
\providecommand{\DIFdelbegin}{} %DIF PREAMBLE
\providecommand{\DIFdelend}{} %DIF PREAMBLE
\providecommand{\DIFmodbegin}{} %DIF PREAMBLE
\providecommand{\DIFmodend}{} %DIF PREAMBLE
%DIF FLOATSAFE PREAMBLE %DIF PREAMBLE
\providecommand{\DIFaddFL}[1]{\DIFadd{#1}} %DIF PREAMBLE
\providecommand{\DIFdelFL}[1]{\DIFdel{#1}} %DIF PREAMBLE
\providecommand{\DIFaddbeginFL}{} %DIF PREAMBLE
\providecommand{\DIFaddendFL}{} %DIF PREAMBLE
\providecommand{\DIFdelbeginFL}{} %DIF PREAMBLE
\providecommand{\DIFdelendFL}{} %DIF PREAMBLE
%DIF HYPERREF PREAMBLE %DIF PREAMBLE
\providecommand{\DIFadd}[1]{\texorpdfstring{\DIFaddtex{#1}}{#1}} %DIF PREAMBLE
\providecommand{\DIFdel}[1]{\texorpdfstring{\DIFdeltex{#1}}{}} %DIF PREAMBLE
\newcommand{\DIFscaledelfig}{0.5}
%DIF HIGHLIGHTGRAPHICS PREAMBLE %DIF PREAMBLE
\RequirePackage{settobox} %DIF PREAMBLE
\RequirePackage{letltxmacro} %DIF PREAMBLE
\newsavebox{\DIFdelgraphicsbox} %DIF PREAMBLE
\newlength{\DIFdelgraphicswidth} %DIF PREAMBLE
\newlength{\DIFdelgraphicsheight} %DIF PREAMBLE
% store original definition of \includegraphics %DIF PREAMBLE
\LetLtxMacro{\DIFOincludegraphics}{\includegraphics} %DIF PREAMBLE
\newcommand{\DIFaddincludegraphics}[2][]{{\color{blue}\fbox{\DIFOincludegraphics[#1]{#2}}}} %DIF PREAMBLE
\newcommand{\DIFdelincludegraphics}[2][]{% %DIF PREAMBLE
\sbox{\DIFdelgraphicsbox}{\DIFOincludegraphics[#1]{#2}}% %DIF PREAMBLE
\settoboxwidth{\DIFdelgraphicswidth}{\DIFdelgraphicsbox} %DIF PREAMBLE
\settoboxtotalheight{\DIFdelgraphicsheight}{\DIFdelgraphicsbox} %DIF PREAMBLE
\scalebox{\DIFscaledelfig}{% %DIF PREAMBLE
\parbox[b]{\DIFdelgraphicswidth}{\usebox{\DIFdelgraphicsbox}\\[-\baselineskip] \rule{\DIFdelgraphicswidth}{0em}}\llap{\resizebox{\DIFdelgraphicswidth}{\DIFdelgraphicsheight}{% %DIF PREAMBLE
\setlength{\unitlength}{\DIFdelgraphicswidth}% %DIF PREAMBLE
\begin{picture}(1,1)% %DIF PREAMBLE
\thicklines\linethickness{2pt} %DIF PREAMBLE
{\color[rgb]{1,0,0}\put(0,0){\framebox(1,1){}}}% %DIF PREAMBLE
{\color[rgb]{1,0,0}\put(0,0){\line( 1,1){1}}}% %DIF PREAMBLE
{\color[rgb]{1,0,0}\put(0,1){\line(1,-1){1}}}% %DIF PREAMBLE
\end{picture}% %DIF PREAMBLE
}\hspace*{3pt}}} %DIF PREAMBLE
} %DIF PREAMBLE
\LetLtxMacro{\DIFOaddbegin}{\DIFaddbegin} %DIF PREAMBLE
\LetLtxMacro{\DIFOaddend}{\DIFaddend} %DIF PREAMBLE
\LetLtxMacro{\DIFOdelbegin}{\DIFdelbegin} %DIF PREAMBLE
\LetLtxMacro{\DIFOdelend}{\DIFdelend} %DIF PREAMBLE
\DeclareRobustCommand{\DIFaddbegin}{\DIFOaddbegin \let\includegraphics\DIFaddincludegraphics} %DIF PREAMBLE
\DeclareRobustCommand{\DIFaddend}{\DIFOaddend \let\includegraphics\DIFOincludegraphics} %DIF PREAMBLE
\DeclareRobustCommand{\DIFdelbegin}{\DIFOdelbegin \let\includegraphics\DIFdelincludegraphics} %DIF PREAMBLE
\DeclareRobustCommand{\DIFdelend}{\DIFOaddend \let\includegraphics\DIFOincludegraphics} %DIF PREAMBLE
\LetLtxMacro{\DIFOaddbeginFL}{\DIFaddbeginFL} %DIF PREAMBLE
\LetLtxMacro{\DIFOaddendFL}{\DIFaddendFL} %DIF PREAMBLE
\LetLtxMacro{\DIFOdelbeginFL}{\DIFdelbeginFL} %DIF PREAMBLE
\LetLtxMacro{\DIFOdelendFL}{\DIFdelendFL} %DIF PREAMBLE
\DeclareRobustCommand{\DIFaddbeginFL}{\DIFOaddbeginFL \let\includegraphics\DIFaddincludegraphics} %DIF PREAMBLE
\DeclareRobustCommand{\DIFaddendFL}{\DIFOaddendFL \let\includegraphics\DIFOincludegraphics} %DIF PREAMBLE
\DeclareRobustCommand{\DIFdelbeginFL}{\DIFOdelbeginFL \let\includegraphics\DIFdelincludegraphics} %DIF PREAMBLE
\DeclareRobustCommand{\DIFdelendFL}{\DIFOaddendFL \let\includegraphics\DIFOincludegraphics} %DIF PREAMBLE
%DIF COLORLISTINGS PREAMBLE %DIF PREAMBLE
\RequirePackage{listings} %DIF PREAMBLE
\RequirePackage{color} %DIF PREAMBLE
\lstdefinelanguage{DIFcode}{ %DIF PREAMBLE
%DIF DIFCODE_UNDERLINE %DIF PREAMBLE
  moredelim=[il][\color{red}\sout]{\%DIF\ <\ }, %DIF PREAMBLE
  moredelim=[il][\color{blue}\uwave]{\%DIF\ >\ } %DIF PREAMBLE
} %DIF PREAMBLE
\lstdefinestyle{DIFverbatimstyle}{ %DIF PREAMBLE
	language=DIFcode, %DIF PREAMBLE
	basicstyle=\ttfamily, %DIF PREAMBLE
	columns=fullflexible, %DIF PREAMBLE
	keepspaces=true %DIF PREAMBLE
} %DIF PREAMBLE
\lstnewenvironment{DIFverbatim}{\lstset{style=DIFverbatimstyle}}{} %DIF PREAMBLE
\lstnewenvironment{DIFverbatim*}{\lstset{style=DIFverbatimstyle,showspaces=true}}{} %DIF PREAMBLE
%DIF END PREAMBLE EXTENSION ADDED BY LATEXDIFF

\begin{document}
\DIFdelbegin %DIFDELCMD < \include{frontmatter_v0}
%DIFDELCMD < \include{introduction_v0}
%DIFDELCMD < \include{kinematics_v0}
%DIFDELCMD < \include{mix_v0}
%DIFDELCMD < \include{boundary_v0}
%DIFDELCMD < \include{examples_v0}
%DIFDELCMD < \include{conclusion_v0}
%DIFDELCMD < %%%
\DIFdelend \DIFaddbegin 
\begin{frontmatter}
    \title{A Hu-Washizu variational consistent meshfree thin shell formulation with naturally accommodating essential boundary conditions}
    \author[1]{Junchao Wu\corref{cor1}}
    \ead{jcwu@hqu.edu.cn}
    \author[2]{Yangtao Xu}
    \author[1]{Bin Xu}
    \author[3]{Syed Humayun Basha}

    \affiliation[1]{organization={Key Laboratory for Intelligent Infrastructure and Monitoring of Fujian Province,
                                  College of Civil Engineering,
                                  Huaqiao University},
                    city={Xiamen},
                    state={Fujian},
                    postcode={361021},
                    country={China}}

    \affiliation[2]{organization={College of Civil Engineering,
                                  Huaqiao University},
                    city={Xiamen},
                    state={Fujian},
                    postcode={361021},
                    country={China}}

    \affiliation[3]{organization={Key Laboratory for Structural Engineering and Disaster Prevention of Fujian Province,
                                  College of Civil Engineering,
                                  Huaqiao University},
                    city={Xiamen},
                    state={Fujian},
                    postcode={361021},
                    country={China}}

    \cortext[cor1]{Corresponding author}

\begin{abstract}
A Hu-Washizu principle based variational consistent meshfree formulation with naturally enforcement of essential boundary conditions is proposed for thin shell analysis. In this approach, a mixed formulation of displacements, strains and stresses within the framework of Hu-Washizu variational principle is employed, where the displacements are discretized by meshfree shape functions, the strains and stresses are expressed as the smoothed gradients and covariant smoothed gradients which meet the first two order integration constraint and have the quasi- varational consistency. 
The numerical examples demonstrate the efficacy and efficiency of proposed method, while the RKGSI performs a comparable result in energy error with interpolation by meshfree approximations.
\end{abstract}
\begin{keyword}
Meshfree \sep Thin shell \sep Hu-Washizu variational principle \sep Reproducing kernel gradient smoothing \sep Essential boundary condition
\end{keyword}
\end{frontmatter}

\section{Introduction}\label{introduction}
Thin shell structures generally adhere to the Kirchhoff hypothesis \cite{donnell1976}, that neglects the shear deformation can be described using Galerkin formulation which requires to have at least $C^1$ continuity. 
\DIFdelbegin \DIFdel{The traditional }\DIFdelend \DIFaddbegin \DIFadd{Traditional }\DIFaddend finite element methods \DIFdelbegin \DIFdel{usually have }\DIFdelend \DIFaddbegin \DIFadd{typically employ }\DIFaddend $C^0$ continuous shape functions, and it prefers \DIFdelbegin \DIFdel{Mindlin thick shear theory, }\DIFdelend hybrid and mixed \DIFdelbegin \DIFdel{models in simulation of shell structure  \mbox{%DIFAUXCMD
\cite{hughes2000}}\hskip0pt%DIFAUXCMD
. 
Meshfree methods \mbox{%DIFAUXCMD
\cite{belytschko1994,liu1995,chen2017} }\hskip0pt%DIFAUXCMD
}\DIFdelend \DIFaddbegin \DIFadd{shell models, like linear and nonlinear Mindlin model \mbox{%DIFAUXCMD
\cite{ahmad1970, hughes2000} }\hskip0pt%DIFAUXCMD
and the one inextensible director model \mbox{%DIFAUXCMD
\cite{simo1989b}}\hskip0pt%DIFAUXCMD
. 
Over the past thirty years, various novel formulations }\DIFaddend with high order smoothed shape functions have \DIFdelbegin \DIFdel{garnered much research attention over the past thirty years. These techniques }\DIFdelend \DIFaddbegin \DIFadd{been applied to thin shell formulations. These include element-free Galerkin method \mbox{%DIFAUXCMD
\cite{krysl1996}}\hskip0pt%DIFAUXCMD
, maximum-entropy meshfree method \mbox{%DIFAUXCMD
\cite{millan2011}}\hskip0pt%DIFAUXCMD
, Hermite reproducing kernel particle method \mbox{%DIFAUXCMD
\cite{wang2015a}}\hskip0pt%DIFAUXCMD
, peridynamics \mbox{%DIFAUXCMD
\cite{behzadinasab2022}}\hskip0pt%DIFAUXCMD
, isogeometric analysis \mbox{%DIFAUXCMD
\cite{kiendl2009}}\hskip0pt%DIFAUXCMD
, and others.
For a more comprehensive review of advances and applications of high order formulations in various scientific and engineering fields, refer to \mbox{%DIFAUXCMD
\cite{liu2009,chen2017,zhang2017a,suchde2022,wang2023b,deng2023a,wang2024,wang2024a}}\hskip0pt%DIFAUXCMD
.
Among these approaches, Galerkin meshfree methods with moving least square approximation (MLS) \mbox{%DIFAUXCMD
\cite{belytschko1994} }\hskip0pt%DIFAUXCMD
or reproducing kernel approximation (RK) \mbox{%DIFAUXCMD
\cite{liu1995} }\hskip0pt%DIFAUXCMD
}\DIFaddend established the shape functions based on a collection of dispersed nodes, and high order continuity of shape functions can be easily achieved even with low-order basis functions. For thin shell analysis, high order meshfree approximation can also \DIFdelbegin \DIFdel{furhter }\DIFdelend \DIFaddbegin \DIFadd{further }\DIFaddend alleviate the membrane locking caused by the mismatched approximation order of membrane strain and bending strain \cite{krysl1996}. 
Moreover, \DIFdelbegin \DIFdel{nodal-based meshfree }\DIFdelend \DIFaddbegin \DIFadd{node-based MLS/RK }\DIFaddend approximations generally offer the flexibility of local refinement and can relieve the burden of mesh distortion. 
\DIFdelbegin \DIFdel{Owing to these benefits, numerous meshfree techniques have been developed and implemented in many scientific and engineering fields \mbox{%DIFAUXCMD
\cite{liu2009,zhang2000,millan2011,wang2023b,suchde2022,deng2023a,wang2024}}\hskip0pt%DIFAUXCMD
. }\DIFdelend However, the high order smoothed meshfree shape functions accompany the enlarged and overlapping supports, which may potentially cause many problems for shape functions. One of the issues is the loss of the Kronecker delta property, which means that, unlike the finite element methods, the necessary boundary conditions cannot be directly enforced  \cite{fernandez-mendez2004}. Another issue is that the variational consistency or said integration constraint, which is a condition that requires the formulation to exactly reproduce the solution spanned by the basis functions, cannot be satisfied. This issue is mainly caused by the misalignment between the numerical integration domains and supports of shape functions. Thus, the shape functions exhibit a piecewise nature in each integration domain. Besides, it has to be noted that the traditional integration rules like Gauss scheme cannot ensure the integration accuracy in Galerkin weak form \cite{li2016, wu2021}. Therefore, variational consistency is vital to the solution accuracy in the Galerkin meshfree formulations.

Various ways have been presented to enforce the necessary boundary for Galerkin meshfree methods directly, including the boundary singular kernel method \cite{chen2000a}, mixed transformation method  \cite{chen2000a}, and interpolation element-free method \cite{liu2019a} for recovering shape functions’ Kronecker property. However, these methods were not based on variational setting and cannot guarantee variational consistency. \DIFdelbegin \DIFdel{In the absence of a meshfree node, accuracy enforcement might be poor }\DIFdelend \DIFaddbegin \DIFadd{The accuracy maybe poor at locations away from the sample points}\DIFaddend . In contrast, enforcing the essential boundary conditions using a variational approach is preferred for Galerkin meshfree methods. The variational consistent Lagrange multiplier approach was initially used to the Galerkin meshfree method by Belytschko et al. \cite{belytschko1994}. In this method, the extra degrees of freedom are used to determine the discretion of Lagrange multiplier. Ivannikov et al. \cite{ivannikov2014a} extended this approach to geometrically nonlinear thin shells. Lu et al. \cite{lu1994} suggested the modified variational essential boundary enforcement approach and expressed the Lagrange multiplier by equivalent \DIFdelbegin \DIFdel{tractions }\DIFdelend \DIFaddbegin \DIFadd{traction }\DIFaddend to eliminate the excess degrees of freedom. However, the coercivity of this approach is not always ensured and potentially reduces the accuracy. Zhu and Atluri \cite{zhu1998} pioneered the penalty method for meshfree method, making it a straightforward approach to enforce essential boundary conditions via Galerkin weak form. However, the penalty method lacks variational consistency and requires experimental artificial parameters whose optimal value is hard to determine. Fernández-Méndez and Huerta \cite{fernandez-mendez2004} imposed necessary boundary conditions using Nitsche's approach in the meshfree formulation. This approach can be seen as a hybrid combination of the modified variational method and the penalty method because the modified variational method generates variational consistency through the use of a consistent term, and the penalty method is used as a stabilized term to recover the coercivity. Skatulla and Sansour \cite{skatulla2008} extended Nitsche’s thin shell analysis method and proposed an iteration algorithm to determine artificial parameters at each integration point\DIFaddbegin \DIFadd{.
Additionally, the Nitsche's method has been successfully applied to maintain the variational consistency between different geometrical or material domains in problems with multiple patches \mbox{%DIFAUXCMD
\cite{guo2021} }\hskip0pt%DIFAUXCMD
and composite materials \mbox{%DIFAUXCMD
\cite{wang2021b}}\hskip0pt%DIFAUXCMD
}\DIFaddend .

In order to address the issue of numerical integration, a series of consistent integration schemes have been developed for Galerkin meshfree methods. Among these include stabilized conforming nodal integration \cite{chen2001}, variational consistent integration \cite{chen2013a}, quadratic consistent integration \cite{duan2012a}, reproducing kernel gradient smoothing integration \cite{wang2019a}, and consistent projection integration \cite{wang2023}. The assumed strain approach establishes the most consistent integration scheme, while the smoothed gradient replaces the costly higher order derivatives of traditional meshfree shape functions and shows a high efficiency. Moreover, to achieve global variational consistency, a consistent essential boundary condition enforcement must be combined with the consistent integration scheme. The combination of consistent integration scheme and Nitsche’s method for treating essential boundary conditions may demonstrate better performance since both the methods can satisfy the coercivity without requiring additional degrees of freedom. Nevertheless, Nitsche's approach still retains the artificial parameters in the stabilized terms, and it is essential to remain cautious of the costly higher order derivatives, particularly for thin plate and thin shell problems. Recently, Wu et al. \cite{wu2022a,wu2023}  proposed an efficient and stabilized essential boundary condition enforcement method based upon the Hellinger-Reissner variational principle, where a mixed formulation in Hellinger-Reissner weak form recasts the reproducing kernel gradient smoothing integration. The terms required for enforcing essential boundary conditions are identical to the Nitsche’s method, and both have consistent and stabilized terms. However, the stabilized term of this method naturally exists in the Hellinger-Reissner weak form and no longer needs the artificial parameters, even for essential boundary enforcement. Instead all of the higher order derivatives are represented by the smoothed gradients and their derivatives.

In this study, an efficient and stabilized variational consistent meshfree method that naturally enforces the essential boundary conditions is developed for thin shell structures. Following the concept of the Hellinger-Reissner principle base consistent meshfree method, the Hu-Washizu variational principle of complementary energy with variables of displacement, strains, and stresses were employed. The displacement is approximated by conventional meshfree shape functions, and the strains and stresses were expressed by smoothed gradients with covariant bases. It is important to note that although the first second-order integration requirements were naturally embedded in the smoothed gradients, their fulfillment resulted in a quasi-satisfaction of variational consistency. This is mainly because of the non-polynomial nature of the stresses. Hu-Washizu's weak form was used to evaluate all the essential boundary conditions regarding displacements and rotations. This type of formulation is similar to the Nitsche's method but does not require any artificial parameters. Compared with Nitsche’s method, conventional reproducing smoothed gradients and its direct derivatives replace the costly higher order derivatives. By utilizing the advantages of a replicating kernel gradient smoothing framework, the smoothed gradients showed better performance compared to conventional derivatives of shape functions, hence increasing the meshfree formulation's computational efficiency.

The remainder of this research article is structured as follows: The kinematics of the thin shell structure and the weak form of the associated Hu-Washizu principle are briefly described in Section 2. The mixed formulation regarding the displacements, strains and stresses in accordance with Hu-Washizu weak form are presented in Section 3. The discrete equilibrium equations are derived in Section 4 using the naturally occurring accommodation of essential. Subsequently, they are compared to the equations obtained using Nitsche's method. The numerical results in Section 5 validate the efficacy of the proposed Hu-Washizu meshfree thin shell formulation. Lastly, the concluding remarks are presented in Section 6.


\section{Hu-Washizu's formulation of complementary energy for thin shell}
\subsection{Kinematics for thin shell}
Consider the configuration of a shell $\bar \Omega$, as shown in Fig. \ref{}, which can be easily described by a parametric curvilinear coordinate system $\boldsymbol \xi = \{\xi^i\}_{i=1,2,3}$. The mid-surface of the shell is specified by the in-plane coordinates $\boldsymbol \xi = \{\xi^\alpha\}_{\alpha=1,2}$, as the thickness direction of shell is by $\xi^3$, $-\frac{h}{2} \le \xi^3 \le \frac{h}{2}$, $h$ is the thickness of shell. In this work, Latin indices take the values from 1 to 3, and Greek indices are evaluated by 1 or 2. For the Kirchhoff hypothesis \cite{krysl1996}, the position $\boldsymbol x\in \bar \Omega$ are defined by linear functions with respect to $\xi^3$ :
\begin{equation}\label{x}
\boldsymbol x(\xi^1, \xi^2, \xi^3) = \boldsymbol r(\xi^1,\xi^2) + \xi^3 \boldsymbol a_3(\xi^1,\xi^2)
\end{equation}
in which $\boldsymbol r$ means the position on the mid-surface of shell, and the $\boldsymbol a_3$ is corresponding normal direction. For the mid-surface of shell, the in-plane covariant base vector with respect to $\xi^\alpha$ can be derived by a trivial partial differentiation to $\boldsymbol r$:
\begin{equation}
\boldsymbol a_\alpha = \frac{\partial \boldsymbol r}{\partial \xi^\alpha} = \boldsymbol r_{,\alpha}, \alpha  = 1,2
\end{equation}
for a clear expression, the subscript comma denotes the partial differentiation operation with respect to in-plane coordinates $\xi^\alpha$. And the normal vector $\boldsymbol a_3$ can be obtained by the normalized cross product of $\boldsymbol a_{\alpha}$'s as follow:
\begin{equation}
\boldsymbol a_3 = \frac{\boldsymbol a_1 \times \boldsymbol a_2}{\Vert \boldsymbol a_1 \times \boldsymbol a_2 \Vert}
\end{equation}
where $\Vert \bullet \Vert$ is the Euclidean norm operator.

With the assumption of infinitesimal deformation, the strain components respected to global contravariant base can be sated as:
\begin{equation}\label{epsilon}
\epsilon_{ij} = \frac{1}{2}(\boldsymbol x_{,i} \cdot \boldsymbol u_{,j} + \boldsymbol u_{,i} \cdot \boldsymbol x_{,j})
\end{equation}
where $\boldsymbol u$ is the displacement for shell deformation. To fulfillment with Kirchhoff hypothesis, the displacement is assumed to be the following form:
\begin{equation}\label{u}
\boldsymbol u(\xi^1,\xi^2,\xi^3) = \boldsymbol v(\xi^1,\xi^2) + \boldsymbol \theta(\xi^1,\xi^2) \xi^3
\end{equation}
in which the quadratic and higher order terms are neglected. $\boldsymbol v$, $\boldsymbol \theta$ respect the displacement and rotation in mid-surface.

Subsequently, plugging Eqs. (\ref{x}) and (\ref{u}) into Eq. (\ref{epsilon}) and neglecting quadratic terms, the strain components can be rephrased as follows:
\begin{subequations}
\begin{align}
\begin{split}
\epsilon_{\alpha\beta} &= \frac{1}{2}(\boldsymbol a_\alpha \cdot \boldsymbol v_{,\beta} + \boldsymbol v_{,\alpha}\cdot \boldsymbol a_\beta) \\ 
&+ \frac{1}{2}(\boldsymbol a_{3,\alpha} \cdot \boldsymbol v_{,\beta} + \boldsymbol v_{,\alpha}\cdot \boldsymbol a_{3,\beta} + \boldsymbol a_\alpha \cdot \boldsymbol \theta_{,\beta} + \boldsymbol \theta_{,\alpha}\cdot \boldsymbol a_\beta)\xi^3 \\
&= \varepsilon_{\alpha\beta} + \kappa_{\alpha\beta}\xi^3
\end{split} \\
\epsilon_{\alpha 3} &= \frac{1}{2}(\boldsymbol a_\alpha \cdot \boldsymbol \theta + \boldsymbol v_{,\alpha}\cdot \boldsymbol a_3) + \frac{1}{2} (\boldsymbol a_3 \cdot \boldsymbol \theta)_{,\alpha}\xi^3 \\ 
\epsilon_{33} &= \boldsymbol a_3 \cdot \boldsymbol \theta
\end{align}
\end{subequations}
where $\varepsilon_{\alpha\beta}$, $\kappa_{\alpha\beta}$ are membrane and bending strains respectively:
\begin{equation}
\varepsilon_{\alpha\beta} = \frac{1}{2}(\boldsymbol a_\alpha \cdot \boldsymbol v_{,\beta} + \boldsymbol v_{,\alpha}\cdot \boldsymbol a_\beta) 
\end{equation}
\begin{equation}\label{kappa1}
\kappa_{\alpha\beta} = \frac{1}{2}(\boldsymbol a_{3,\alpha} \cdot \boldsymbol v_{,\beta} + \boldsymbol v_{,\alpha}\cdot \boldsymbol a_{3,\beta} + \boldsymbol a_\alpha \cdot \boldsymbol \theta_{,\beta} + \boldsymbol \theta_{,\alpha}\cdot \boldsymbol a_\beta)
\end{equation}

In accordance with Kirchhoff hypothesis, the thickness of shell will not change and the deformation related with direction of $\xi^3$ will be vanished, i.e. $\epsilon_{3i}=0$. Thus, the rotation $\boldsymbol \theta$ can be rewritten as:
\begin{equation}\label{a3}
\epsilon_{3i} = 0 \Rightarrow
\left \{
\begin{split}
&\boldsymbol \theta \cdot \boldsymbol a_\alpha + \boldsymbol v_{,\alpha} \cdot \boldsymbol a_3 = 0 \\
&\boldsymbol \theta \cdot \boldsymbol a_3 = 0
\end{split}
\right .
\Rightarrow \boldsymbol \theta = - \boldsymbol v_{,\alpha} \cdot \boldsymbol a_3 \boldsymbol a^\alpha
\end{equation}
where $\boldsymbol a^\alpha$'s are the in-plane contravariant base vectors, $\boldsymbol a^\alpha \cdot \boldsymbol a_\beta = \delta^\alpha_\beta$, $\delta$ is the Kronecker delta function. Substituting Eq. (\ref{a3}) into Eq. (\ref{kappa1}) leads to:
\begin{equation}
\kappa_{\alpha\beta} = (\Gamma^\gamma_{\alpha\beta} \boldsymbol v_{,\gamma} - \boldsymbol v_{,\alpha\beta}) \cdot \boldsymbol a_3 = - \boldsymbol v_{,\alpha}\vert_\beta \cdot \boldsymbol a_3
\end{equation}
in which $\Gamma^\gamma_{\alpha\beta} = \boldsymbol a_{\alpha,\beta} \cdot \boldsymbol a^\gamma$ is namely Christoffel symbol of the second kind.

\section{Mixed meshfree formulation for modified Hellinger-Reissner weak form}
\subsection{Reproducing kernel approximation for displacement}
In this study, the displacement is approximated by traditional reproducing kernel approximation. As shown in Fig, 

\begin{equation}
\boldsymbol v(\boldsymbol x) = \sum_{I=1}^{n_p} \Psi_I(\boldsymbol x) \boldsymbol d_I
\end{equation}

\begin{equation}
    \boldsymbol v_{,\alpha}(\boldsymbol x) = \boldsymbol p^T(\boldsymbol x) \boldsymbol d_{\alpha}^\varepsilon, \quad
    \varepsilon_{\alpha\beta}(\boldsymbol x) = \boldsymbol p^T(\boldsymbol x) \frac{1}{2}(\boldsymbol a_\alpha \cdot \boldsymbol d_{\beta}^\varepsilon + \boldsymbol a_\beta \cdot \boldsymbol d_{\alpha}^\varepsilon)
\end{equation}

\subsection{Reproducing kernel gradient smoothing approximation for effective stress and strain}
For the inspiration 
\begin{equation}
    -(\boldsymbol v_{,\alpha})\vert_\beta(\boldsymbol x) = \boldsymbol p^T(\boldsymbol x) \boldsymbol d_{\alpha\beta}^\kappa , \quad
    \kappa^{\alpha\beta}(\boldsymbol x) = - \boldsymbol p^T(\boldsymbol x) \boldsymbol a_3 \cdot \boldsymbol d_{\alpha\beta}^\kappa
\end{equation}
\begin{equation}
N^{\alpha\beta}(\boldsymbol x) = \boldsymbol p^T(\boldsymbol x) \boldsymbol a^\alpha \cdot \boldsymbol d_{\beta}^N,\quad
\boldsymbol a_\alpha N^{\alpha\beta} = \boldsymbol p^T(\boldsymbol x) \boldsymbol d^N_\beta
\end{equation}
\begin{equation}
    M^{\alpha\beta}(\boldsymbol x) = \boldsymbol p^T(\boldsymbol x) \boldsymbol a_3 \cdot \boldsymbol d_{\alpha\beta}^M,\quad
    \boldsymbol a_3 M^{\alpha\beta} = \boldsymbol p^T(\boldsymbol x) \boldsymbol d_{\alpha\beta}^M
\end{equation}

\section{Naturally variational enforcement for essential boundary conditions}


\section{Numerical examples}\label{examples}
In this section, the suggested method is validated through several examples using the Nitsche's method, the consistent reproducing kernel gradient smoothing integration scheme (RKGSI), and the non-consistent Gauss integration scheme (GI) with penalty method, as well as the proposed Hu-Washizu formulation (HW) to enforce the necessary boundary conditions. A normalized support size of 2.5 is used for all the considered methods to ensure the requirement of quadratic base meshfree approximation. To eliminate the influence of integration error, the Gauss integration scheme uses 6 Gauss points for domain integration and 3 points for boundary integration, so as to maintain the same integration accuracy between domain and boundaries. Moreover, the number of integration points are identical between the Gauss and RKGSI schemes. The error estimates of displacement ($L_2$-Error) and energy ($H_e$-Error) is used here:
\begin{equation}
\begin{split}
L_2\text{-Error} &= \frac{\sqrt{\int_\Omega(\boldsymbol v - \boldsymbol v^h) \cdot (\boldsymbol v - \boldsymbol v^h)d\Omega}}{\sqrt{\boldsymbol v \cdot \boldsymbol v}} \\
H_e\text{-Error} &= \frac{\sqrt{\int_\Omega \left ((\varepsilon_{\alpha\beta} - \varepsilon_{\alpha\beta}^h)(N^{\alpha\beta} - N^{\alpha\beta h}) + \int_\Omega(\kappa_{\alpha\beta}-\kappa_{\alpha\beta}^h)(M^{\alpha\beta}-M^{\alpha\beta h}) \right )d\Omega}}{\sqrt{\int_\Omega(\varepsilon_{\alpha\beta}N^{\alpha\beta} + \kappa_{\alpha\beta}M^{\alpha\beta})d\Omega}}
\end{split}
\end{equation}

\subsection{Patch tests}
The linear and quadratic patch tests for flat and curved thin shells are firstly studied to verify the variational consistency of the proposed method. As shown in Fig. \ref{ptf1}, the flat and curved models are depicted by an identical parametric domain $\Omega = (0,1)\otimes(0,1)$, where the cylindrical coordinate system with radius $R=1$, thickness \DIFdelbegin \DIFdel{$h=0.1$ }\DIFdelend \DIFaddbegin \DIFadd{$h=0.05$ }\DIFaddend is employed to describe the curved model, and the whole domain $\Omega$ is discretized by the $165$ meshfree nodes. The Young's modulus and Poisson's ratio of thin shell are set to $E=1$, $\nu=0$. The artificial parameters of $\alpha_v=10^5\times E$, $\alpha_\theta=10^3\times E$, $\alpha_C=10^5\times E$ and $\alpha_v=10^9\times E, \alpha_\theta=10^9\times E, \alpha_C=10^9\times E$ were adopted in Nitsche's- and penalty- method, respectively. All the boundaries are enforced as essential boundary conditions with the following manufactured exact solution:
\begin{equation}
\boldsymbol v = \begin{Bmatrix}
(\xi^1+2\xi^2)^n \\ (3\xi^1+4\xi^2)^n \\ (5\xi^1+6\xi^2)^n
\end{Bmatrix},\quad
n = \begin{cases}
1 & \text{Linear patch test} \\
2 & \text{Quadratic patch test}
\end{cases}
\end{equation}

\begin{figure}[!ht]
    \centering
    \includegraphics[width=\textwidth]{figures/ptmsh}
    \caption{Meshfree discretization for patch test}\label{ptf1}
\end{figure}

Table \ref{ptt1} lists the $L_2$- and $H_e$-Error results of patch test with flat model, where the RKGSI scheme with variational consistent essential boundary enforcement, i.e. RKGSI-Nitsche and RKGSI-HW, can pass the linear and quadratic patch test. In contrast, the RKGSI-Penalty cannot pass the patch test since the Penalty method is unable to ensure the variational consistency. Due to the loss of variational consistency condition, even with the Nitsche's method, Gauss meshfree formulations show noticeable errors. Table \ref{ptt2} shows the results for curved model, which indicated that all the considered methods cannot pass the patch test. This is mainly because the proposed smoothed gradient of Eqs. (\ref{approxse1}) and (\ref{approxse2}) could not exactly reproduce the non-polynomial membrane and bending stresses. On the other hand, the RKGSI-HW and RKGSI-Nitsche methods provide better accuracy compared to the other approaches due to the fulfillment of first second-order variational consistency. Even only with local variational consistency, the RKGSI-Penalty obtained a better result than the traditional Gauss scheme. Meanwhile, the bending moment contours of $M^{12}$ are listed in Fig. \ref{ptf2}, which further verify that the proposed method provided a satisfactory result compared to the exact solution. Contrarily, both the RKGSI-Penalty and the conventional Gauss meshree formulations observed errors.

\begin{table}[!ht]
\centering
\caption{Results of patch test for flat model.}\label{ptt1}
\begin{tabular}{lcccc}
\toprule
 & \multicolumn{2}{c}{Linear patch test} & \multicolumn{2}{c}{Quadratic patch test} \\ \cline{2-5}
 & $L_2$-Error & $H_e$-Error & $L_2$-Error & $H_e$-Error \\
    \midrule
    GI-Penalty & \DIFdelbeginFL \DIFdelFL{4.45}\DIFdelendFL \DIFaddbeginFL \DIFaddFL{1.41}\DIFaddendFL E-04 & \DIFdelbeginFL \DIFdelFL{1.35E-02 }\DIFdelendFL \DIFaddbeginFL \DIFaddFL{4.62E-03 }\DIFaddendFL & \DIFdelbeginFL \DIFdelFL{2.01}\DIFdelendFL \DIFaddbeginFL \DIFaddFL{1.97}\DIFaddendFL E-03 & \DIFdelbeginFL \DIFdelFL{1.63E-02 }\DIFdelendFL \DIFaddbeginFL \DIFaddFL{7.17E-03 }\DIFaddendFL \\
    GI-Nitsche & \DIFdelbeginFL \DIFdelFL{4.51}\DIFdelendFL \DIFaddbeginFL \DIFaddFL{1.73}\DIFaddendFL E-04 & \DIFdelbeginFL \DIFdelFL{1.42E-02 }\DIFdelendFL \DIFaddbeginFL \DIFaddFL{5.61E-03 }\DIFaddendFL & \DIFdelbeginFL \DIFdelFL{1.22}\DIFdelendFL \DIFaddbeginFL \DIFaddFL{1.85}\DIFaddendFL E-03 & \DIFdelbeginFL \DIFdelFL{1.68E-02 }\DIFdelendFL \DIFaddbeginFL \DIFaddFL{7.76E-03 }\DIFaddendFL \\
    RKGSI-Penalty & \DIFdelbeginFL \DIFdelFL{3.64}\DIFdelendFL \DIFaddbeginFL \DIFaddFL{5.04}\DIFaddendFL E-09 & \DIFdelbeginFL \DIFdelFL{6.77E-08 }\DIFdelendFL \DIFaddbeginFL \DIFaddFL{1.02E-07 }\DIFaddendFL & \DIFdelbeginFL \DIFdelFL{4.54}\DIFdelendFL \DIFaddbeginFL \DIFaddFL{3.01}\DIFaddendFL E-09 & \DIFdelbeginFL \DIFdelFL{6.57}\DIFdelendFL \DIFaddbeginFL \DIFaddFL{3.41}\DIFaddendFL E-08 \\
    RKGSI-Nitsche & \DIFdelbeginFL \DIFdelFL{3.31}\DIFdelendFL \DIFaddbeginFL \DIFaddFL{9.75}\DIFaddendFL E-12 & \DIFdelbeginFL \DIFdelFL{1.34}\DIFdelendFL \DIFaddbeginFL \DIFaddFL{8.98}\DIFaddendFL E-11 & \DIFdelbeginFL \DIFdelFL{5.98}\DIFdelendFL \DIFaddbeginFL \DIFaddFL{1.29}\DIFaddendFL E-12 & \DIFdelbeginFL \DIFdelFL{1.21E-11 }\DIFdelendFL \DIFaddbeginFL \DIFaddFL{1.06E-12 }\DIFaddendFL \\
    RKGSI-HR & \DIFdelbeginFL \DIFdelFL{6.67}\DIFdelendFL \DIFaddbeginFL \DIFaddFL{6.15}\DIFaddendFL E-13 & \DIFdelbeginFL \DIFdelFL{1.50E-11 }%DIFDELCMD < & %%%
\DIFdelFL{1.07}\DIFdelendFL \DIFaddbeginFL \DIFaddFL{6.91}\DIFaddendFL E-12 & \DIFdelbeginFL \DIFdelFL{1.26E-11 }\DIFdelendFL \DIFaddbeginFL \DIFaddFL{7.51E-13 }& \DIFaddFL{8.36E-12 }\DIFaddendFL \\
    \bottomrule
\end{tabular}
\end{table}

\begin{table}[!ht]
\centering
\caption{Results of patch test for cylindrical model.}\label{ptt2}
\begin{tabular}{lcccc}
\toprule
 & \multicolumn{2}{c}{Linear patch test} & \multicolumn{2}{c}{Quadratic patch test} \\ \cline{2-5}
 & $L_2$-Error & $H_e$-Error & $L_2$-Error & $H_e$-Error \\
    \midrule
    GI-Penalty & \DIFdelbeginFL \DIFdelFL{3.79}\DIFdelendFL \DIFaddbeginFL \DIFaddFL{1.75}\DIFaddendFL E-04 & \DIFdelbeginFL \DIFdelFL{1.30E-02 }\DIFdelendFL \DIFaddbeginFL \DIFaddFL{4.50E-03 }\DIFaddendFL & \DIFdelbeginFL \DIFdelFL{1.74}\DIFdelendFL \DIFaddbeginFL \DIFaddFL{1.08}\DIFaddendFL E-03 & \DIFdelbeginFL \DIFdelFL{1.37E-02 }\DIFdelendFL \DIFaddbeginFL \DIFaddFL{5.83E-03 }\DIFaddendFL \\
    GI-Nitsche & \DIFdelbeginFL \DIFdelFL{4.04}\DIFdelendFL \DIFaddbeginFL \DIFaddFL{1.77}\DIFaddendFL E-04 & \DIFdelbeginFL \DIFdelFL{1.42E-02 }\DIFdelendFL \DIFaddbeginFL \DIFaddFL{5.36E-03 }\DIFaddendFL & \DIFdelbeginFL \DIFdelFL{1.15}\DIFdelendFL \DIFaddbeginFL \DIFaddFL{1.07}\DIFaddendFL E-03 & \DIFdelbeginFL \DIFdelFL{1.49E-02 }\DIFdelendFL \DIFaddbeginFL \DIFaddFL{6.33E-03 }\DIFaddendFL \\
    RKGSI-Penalty & \DIFdelbeginFL \DIFdelFL{1.47}\DIFdelendFL \DIFaddbeginFL \DIFaddFL{8.59E-05 }& \DIFaddFL{9.11}\DIFaddendFL E-04 & \DIFdelbeginFL \DIFdelFL{5.39E-03 }%DIFDELCMD < & %%%
\DIFdelFL{2.26}\DIFdelendFL \DIFaddbeginFL \DIFaddFL{4.28}\DIFaddendFL E-04 & \DIFdelbeginFL \DIFdelFL{2.09}\DIFdelendFL \DIFaddbeginFL \DIFaddFL{2.08}\DIFaddendFL E-03 \\
    RKGSI-Nitsche & \DIFdelbeginFL \DIFdelFL{2.41E-06 }\DIFdelendFL \DIFaddbeginFL \DIFaddFL{1.27E-05 }\DIFaddendFL & \DIFdelbeginFL \DIFdelFL{7.37}\DIFdelendFL \DIFaddbeginFL \DIFaddFL{5.32}\DIFaddendFL E-05 & \DIFdelbeginFL \DIFdelFL{2.47E-06 }%DIFDELCMD < & %%%
\DIFdelFL{2.89}\DIFdelendFL \DIFaddbeginFL \DIFaddFL{1.88}\DIFaddendFL E-05 \DIFaddbeginFL & \DIFaddFL{5.6E-04 }\DIFaddendFL \\
    RKGSI-HR & \DIFdelbeginFL \DIFdelFL{4.28E-06 }\DIFdelendFL \DIFaddbeginFL \DIFaddFL{1.43E-05 }\DIFaddendFL & \DIFdelbeginFL \DIFdelFL{1.30}\DIFdelendFL \DIFaddbeginFL \DIFaddFL{1.60}\DIFaddendFL E-04 & \DIFdelbeginFL \DIFdelFL{9.69E-06 }\DIFdelendFL \DIFaddbeginFL \DIFaddFL{2.93E-05 }\DIFaddendFL & \DIFdelbeginFL \DIFdelFL{2.41}\DIFdelendFL \DIFaddbeginFL \DIFaddFL{2.85}\DIFaddendFL E-04 \\
    \bottomrule
\end{tabular}
\end{table}

\begin{figure}[!ht]
\centering
\DIFdelbeginFL %DIFDELCMD < \includegraphics[width=\textwidth]{figures/ptc}
%DIFDELCMD < %%%
\DIFdelendFL \DIFaddbeginFL \includegraphics[width=\textwidth]{figures/ptc_r2}
\DIFaddendFL \caption{Contour plots of $M^{12}$ for curved shell patch test.}\label{ptf2}
\end{figure}

\subsection{Scordelis-Lo roof}
This example considers the classical Scordelis-Lo roof problem, as depicted in Fig. \ref{slf1}. The cylindrical roof has dimensions $R=25$, $L=50$, $h=0.25$, Young's modulus $E=4.32\times 10^8$ and Poisson's ratio $\nu=0.0$. The entire roof is subjected to \DIFdelbegin \DIFdel{an }\DIFdelend \DIFaddbegin \DIFadd{a }\DIFaddend uniform body force of $b_z = -90$, with the straight edges \DIFdelbegin \DIFdel{remainning }\DIFdelend \DIFaddbegin \DIFadd{remaining }\DIFaddend free and the \DIFdelbegin \DIFdel{the }\DIFdelend curved edges are enforced by $v_x=v_z=0$.

Due to the symmetry, only a quadrant of the model is considered for meshfree analysis, which is discretized by the $11\times 16$, $13\times 20$, $17\times 24$ and $19\times28$ meshfree nodes, as listed in Fig. \ref{slf2}. The comparison of the displacement in $z-$direction at node $A$, $v_{A3}$, is used as the investigated quantity, with the reference value \DIFdelbegin \DIFdel{0.3006 }\DIFdelend \DIFaddbegin \DIFadd{-0.3006 }\DIFaddend given by \cite{kiendl2009}. Firstly, Fig. \ref{slf3} presents a sensitivity study for the artificial parameters of $\alpha_{vi}$'s and $\alpha_\theta$'s in the RKGSI meshfree formulations with the Nitsche's- and penalty- method, where all \DIFdelbegin \DIFdel{of }\DIFdelend the parameters are scaled by the support size as, $\alpha_{v\alpha} = s^{-1}\bar \alpha_v$, $\alpha_{v3} = s^{-3} \bar \alpha_v$ and $\alpha_\theta = s^{-1}\bar \alpha_\theta$. For a better comparison, the result of the proposed RKGSI-HW is also listed in this figure. The results of Fig. \ref{slf3} revealed, that Nitsche's method observed less artificial sensitivity. However, both the methods cannot trivially determine the optimal values of the artificial parameters. The optimal artificial parameters from Fig. \ref{slf3} are adopted for the convergence study in Fig. \ref{slf4}. The convergence result showed that the RKGSI method get satisfactory results while the traditional Gauss methods demonstrated noticeable errors.

\begin{figure}[!ht]
\centering
\includegraphics[width=0.7\textwidth]{figures/slm}
\caption{Description of Scordelis-Lo roof problem.}\label{slf1}
\end{figure}
\begin{figure}[!ht]
\centering
\includegraphics[width=\textwidth]{figures/slmsh}
\caption{Meshfree discretizations for Scordelis-Lo roof problem.}\label{slf2}
\end{figure}
\begin{figure}[!ht]
\centering
\includegraphics[width=\textwidth]{figures/sla_r1}
\caption{Sensitivity comparison of $\alpha_v$ and $\alpha_\theta$ for Scordelis-Lo problem.}\label{slf3}
\end{figure}
\begin{figure}[!ht]
\centering
\DIFdelbeginFL %DIFDELCMD < \includegraphics[width=\textwidth]{figures/sld_r1}
%DIFDELCMD < %%%
\DIFdelendFL \DIFaddbeginFL \includegraphics[width=\textwidth]{figures/sld_r2}
\DIFaddendFL \caption{Displacement convergence for Scordelis-Lo roof problem.}\label{slf4}
\end{figure}

\subsection{Pinched Hemispherical shell}
Consider the hemispherical shell shown in Fig. \ref{phf1}, which is loaded at four points $P=\pm 2$ at $90^\circ$ interval at its bottom. The hemispherical shell has \DIFdelbegin \DIFdel{an }\DIFdelend \DIFaddbegin \DIFadd{a }\DIFaddend radius $R=10$, thickness $h=0.04$, Young's modulus $E=6.825\times10^7$ and Poisson's ratio $\nu = 0.3$.

Due to symmetry, only quadrant model, where the $16\times16$, $24\times24$, $32\times32$ and $40\times40$ meshfree nodes have been discretized as shown in Fig. (\ref{phfm}), were considered. The quantity under investigation for convergence is the displacement at $x$-direction on point $A$, $v_{A1}$ = 0.094 \cite{macneal1985}.
Fig. \ref{phf2} displays the corresponding convergence results, indicating the RKGSI scheme performed significantly better compared to the GI meshfree formulation. Meanwhile, the efficiency comparison for this problem is also shown in Fig. \ref{phf3}, in which the CPU time for assembly and calculation of shape functions are considered. Fig. \ref{phf3}(a) indicates that the RKGSI scheme observed high efficiency in assembly. This is due to the variational inconsistent Gauss meshfree formulation which require more Gaussian points to get satisfactory results. Fig. \ref{phf3}(b) lists the CPU time spent on enforcing essential boundary conditions for the penalty method, Nitsche's method and proposed HW method. The results highlighted that the proposed HW method consumed comparable CPU time in assembly compared to Nitsche's method. However, less time was spent to calculate the shape functions. Since both the HW method and penalty method were developed considering the shape functions first order derivatives. For this reason, both the methods shared an almost identical time in computing the shape functions.
\begin{figure}[!ht]
\centering
\includegraphics[width=0.8\textwidth]{figures/pfm}
\caption{Description of pinched hemispherical shell problem.}\label{phf1}
\end{figure}
\begin{figure}[!ht]
\centering
\includegraphics[width=\textwidth]{figures/pfmsh_r1}
\caption{Meshfree discretizations for pinched hemispherical shell problem.}\label{phfm}
\end{figure}
\begin{figure}[!ht]
\centering
\includegraphics[width=\textwidth]{figures/pfd_r1}
\caption{Displacement convergence for pinched hemispherical shell problem.}\label{phf2}
\end{figure}
\begin{figure}[!ht]
\centering
\includegraphics[width=\textwidth]{figures/efficient}
\caption{Efficiency comparison for pinched hemispherical shell problem: (a) Whole domain; (b) Essential boundaries}\label{phf3}
\end{figure}



\section{Conclusion}\label{conclusion}
An efficient and quasi-consistent meshfree thin shell formulation was presented to naturally enforce the essential boundary conditions. In this approach, the mixed formulation with Hu-Washizu principle weak form is employed, where the displacement is discretized by traditional meshfree shape functions, the strains and stresses can be expressed by reproducing kernel smoothed gradients and covariant smoothed gradients. The smoothed gradient naturally embed the first two order integration constraint, and has a quasi variational consistency for curved models in each integration cells. Owing to the Hu-Washizu variational principle, the essential boundary condition enforcement has a similar form with conventional Nitsche's method, both have the consistent term and stabilized term. Compared with Nitsche's method, the costly high order derivatives in Nitsche's consistent term have been replaced by smoothed gradients, which shows great computational speed due to the reproducing kernel gradient smoothing framework. Meanwhile, the stabilized term is naturally existed in Hu-Washizu weak form, and the artificial parameter needed in Nitsche's stabilized term has been vanished, which can automatically maintain the coercivity for stiffness matrix. Numerical results demonstrated that the proposed Hu-Washizu quasi-consistent meshfree thin shell formulation show great performance in terms of accuracy, efficiency and stability 


\DIFaddend \section*{Acknowledgment}
The support of this work by the National Natural Science Foundation of China (12102138, 52350410467) and the Natural Science Foundation of Fujian Province of China (2023J01108, 2022J05056) is gratefully acknowledged.
\appendix
\section{Covariant derivatives}\label{appderivative}
This Appendix lists the covariant derivatives needed for the method developed in this study.
\begin{equation}
v_{\alpha}\vert_\beta
\end{equation}


\section{Derivations for stiffness metrics and force vectors}
This Appendix details the derivations of stiffness


\begin{thebibliography}{10}
\expandafter\ifx\csname url\endcsname\relax
  \def\url#1{\texttt{#1}}\fi
\expandafter\ifx\csname urlprefix\endcsname\relax\def\urlprefix{URL }\fi
\expandafter\ifx\csname href\endcsname\relax
  \def\href#1#2{#2} \def\path#1{#1}\fi

\bibitem{donnell1976}
L.~H. Donnell, Beams, {{Plates}} and {{Shells}}, McGraw-Hill, 1976.

\DIFaddbegin \bibitem{ahmad1970}
\DIFadd{S.~Ahmad, B.~M. Irons, O.~C. Zienkiewicz, Analysis of thick and thin shell structures by curved finite elements, International Journal for Numerical Methods in Engineering 2 (1970) 419--451.
}

\DIFaddend \bibitem{hughes2000}
T.~J. Hughes, The {{Finite Element Method}}: {{Linear Static}} and {{Dynamic Finite Element Analysis}}, Dover Publications, Mineola, New York, 2000.

\DIFdelbegin
\DIFdel{T.~Belytschko, Y.~Y. Lu, L. ~Gu, Element-free Galerkin methods, International Journal for Numerical Methods in Engineering 37 (1994) 229--256.}\DIFdelend \DIFaddbegin \bibitem{simo1989b}
\DIFadd{J.~C. Simo, D.~D. Fox, On a stress resultant geometrically exact shell model. Part I: Formulation and optimal parametrization, Computer Methods in Applied Mechanics and Engineering 72 (1989) 267--304.}\DIFaddend

\DIFdelbegin
\DIFdel{W.~K.Liu, S.
~Jun, Y.~F.Zhang, Reproducing kernel particle methods, International Journal for Numerical Methods in Fluids 20 (1995) 1081--1106.}\DIFdelend 
\DIFaddbegin \bibitem{krysl1996}
\DIFadd{P.~Krysl, T.~Belytschko, Analysis of thin shells by the }{{\DIFadd{Element-Free Galerkin}}} \DIFadd{method, International Journal of Solids and Structures 33 (1996) 3057--3080.}

\bibitem{millan2011}
\DIFadd{D.~Mill}{\DIFadd{\'a}}\DIFadd{n, A.~Rosolen, M.~Arroyo, Thin shell analysis from scattered points with maximum-entropy approximants, International Journal for Numerical Methods in Engineering 85 (2011) 723--751.}\DIFaddend .

\DIFdelbegin
\DIFdel{J.~S. Chen, M.~Hillman, S.~W. Chi, Meshfree methods: Progress made after 20 Years, Journal of Engineering Mechanics 143 (2017) 04017001.}\DIFdelend 

\DIFaddbegin \bibitem{wang2015a}
\DIFadd{D.~Wang, C.~Song, H.~Peng, A circumferentially enhanced Hermite reproducing kernel meshfree method for buckling analysis of Kirchhoff--Love cylindrical shells, International Journal of Structural Stability and Dynamics 15 (2015) 1450090.}\DIFaddend 

\DIFdelbegin
\DIFdel{P.~Krysl, T.~Belytschko, Analysis of thin shells by the Element-Free Galerkin method, International Journal of Solids and Structures 33 (1996) 3057--3080.}\DIFdelend 
\DIFaddbegin \bibitem{behzadinasab2022}
\DIFadd{M.~Behzadinasab, M.~Alaydin, N.~Trask, Y.~Bazilevs, A general-purpose, inelastic, rotation-free Kirchhoff--Love shell formulation for peridynamics, Computer Methods in Applied Mechanics and Engineering 389 (2022) 114422.}

\bibitem{kiendl2009}
\DIFadd{J.~Kiendl, K.~U. Bletzinger, J.~Linhard, R.~W}{\DIFadd{\"u}}\DIFadd{chner, Isogeometric shell analysis with }{{\DIFadd{Kirchhoff}}}\DIFadd{--}{{\DIFadd{Love}}} \DIFadd{elements, Computer Methods in Applied Mechanics and Engineering 198 (2009) 3902--3914}\DIFaddend .

\bibitem{liu2009}
G.~R. Liu, Meshfree {{Methods}}: {{Moving Beyond}} the {{Finite Element Method}}, {{Second Edition}}, Crc Press, 2009.

\DIFdelbegin \bibitem{zhang2000}
\DIFdel{X.~Zhang, K.~Z. Song, M.~W. Lu, X.~Liu, Meshless methods based on collocation with radial basis functions, Computational Mechanics 26 (2000) 333--343.}\DIFdelend 
\DIFaddbegin \bibitem{chen2017}
\DIFadd{J.~S. Chen, M.~Hillman, S.~W. Chi, Meshfree methods: Progress made after 20 Years, Journal of Engineering Mechanics 143 (2017) 04017001.}\DIFaddend 

\DIFdelbegin
\DIFdel{D.~Mill\'an, A.~Rosolen, M.~Arroyo, Thin shell analysis from scattered points with maximum-entropy approximants, International Journal for Numerical Methodsin Engineering 85 (2011) 723--751.}\DIFdelend 
\DIFaddbegin \bibitem{zhang2017a}
\DIFadd{X.~Zhang, Z.~Chen, Y.~Liu, The Material Point Method: A Continuum-Based Particle Method for Extreme Loading Cases, Academic Press, Oxford, 2017.}

\bibitem{suchde2022}
\DIFadd{P.~Suchde, T.~Jacquemin, O.~Davydov, Point Cloud Generation for Meshfree Methods: An Overview, Archives of Computational Methods in Engineering 30 (2022) 889--915.}\DIFaddend 

\bibitem{wang2023b}
L.~Wang, M.~Hu, Z.~Zhong, F.~Yang, Stabilized {{Lagrange Interpolation Collocation Method}}: {{A}} meshfree method incorporating the advantages of finite element method, Computer Methods in Applied Mechanics and Engineering 404 (2023) 115780.

\DIFdelbegin
\DIFdel{P.~Suchde, T.~Jacquemin, O.~Davydov, Point }%DIFDELCMD < {{%%%
\DIFdel{Cloud Generation}%DIFDELCMD < }} %%%
\DIFdel{for }%DIFDELCMD < {{%%%
\DIFdel{Meshfree Methods}%DIFDELCMD < }}%%%
\DIFdel{: }%DIFDELCMD < {{%%%
\DIFdel{An Overview}%DIFDELCMD < }}%%%
\DIFdel{, Archives of Computational Methods in Engineering 30 (2022) 889--915.
}%DIFDELCMD < 

%DIFDELCMD < %%%
\DIFdelend \bibitem{deng2023a}
L.~Deng, D.~Wang, An accuracy analysis framework for meshfree collocation methods with particular emphasis on boundary effects, Computer Methods in Applied Mechanics and Engineering 404 (2023) 115782.

\bibitem{wang2024}
J.~Wang, M.~Hillman, Upwind reproducing kernel collocation method for convection-dominated problems, Computer Methods in Applied Mechanics and Engineering 420 (2024) 116711.

\DIFaddbegin 
\bibitem{wang2024a}
\DIFadd{J.~Wang, M.~Behzadinasab, W.~Li, Y.~Bazilevs, A stable formulation of
  correspondence-based peridynamics with a computational structure of a method
  using nodal integration, International Journal for Numerical Methods in
  Engineering (2024) e7465.
} \DIFaddend

\DIFaddbegin \bibitem{belytschko1994}
\DIFadd{T.~Belytschko, Y.~Y. Lu, L.~Gu, Element-free }{{\DIFadd{Galerkin}}} \DIFadd{methods, International Journal for Numerical Methods in Engineering 37 (1994) 229--256.
}

\bibitem{liu1995}
\DIFadd{W.~K. Liu, S.~Jun, Y.~F. Zhang, Reproducing kernel particle methods, International Journal for Numerical Methods in Fluids 20 (1995) 1081--1106.
}

\DIFaddend \bibitem{fernandez-mendez2004}
S.~{Fern{\'a}ndez-M{\'e}ndez}, A.~Huerta, Imposing essential boundary conditions in mesh-free methods, Computer Methods in Applied Mechanics and Engineering 193 (2004) 1257--1275.

\bibitem{li2016}
X.~Li, Error estimates for the moving least-square approximation and the element-free {{Galerkin}} method in n-dimensional spaces, Applied Numerical Mathematics 99 (2016) 77--97.

\bibitem{wu2021}
J.~Wu, D.~Wang, An accuracy analysis of {{Galerkin}} meshfree methods accounting for numerical integration, Computer Methods in Applied Mechanics and Engineering 375 (2021) 113631.

\bibitem{chen2000a}
J.~S. Chen, H.~P. Wang, New boundary condition treatments in meshfree computation of contact problems, Computer Methods in Applied Mechanics and Engineering 187 (2000) 441--468.

\bibitem{liu2019a}
D.~Liu, Y.~M. Cheng, The interpolating element-free {{Galerkin}} ({{IEFG}}) method for three-dimensional potential problems, Engineering Analysis with Boundary Elements 108 (2019) 115--123.

\bibitem{ivannikov2014a}
V.~Ivannikov, C.~Tiago, P.~M. Pimenta, On the boundary conditions of the geometrically nonlinear {{Kirchhoff}}--{{Love}} shell theory, International Journal of Solids and Structures 51 (2014) 3101--3112.

\bibitem{lu1994}
Y.~Y. Lu, T.~Belytschko, L.~Gu, A new implementation of the element free {{Galerkin}} method, Computer Methods in Applied Mechanics and Engineering 113 (1994) 397--414.

\bibitem{zhu1998}
T.~Zhu, S.~N. Atluri, A modified collocation method and a penalty formulation for enforcing the essential boundary conditions in the element free {{Galerkin}} method, Computational Mechanics 21 (1998) 211--222.

\bibitem{skatulla2008}
S.~Skatulla, C.~Sansour, Essential boundary conditions in meshfree methods via a modified variational principle: {{Applications}} to shell computations, Computer Assisted Mechanics and Engineering Sciences 15 (2008) 123--142.

\DIFaddbegin \bibitem{guo2021}
\DIFadd{Y.~Guo, Z.~Zou, M.~Ruess, Isogeometric multi-patch analyses for mixed thin shells in the framework of non-linear elasticity, Computer Methods in Applied Mechanics and Engineering 380 (2021) 113771.
}

\bibitem{wang2021b}
\DIFadd{J.~Wang, G.~Zhou, M.~Hillman, A.~Madra, Y.~Bazilevs, J.~Du, K.~Su, Consistent immersed volumetric }{{\DIFadd{Nitsche}}} \DIFadd{methods for composite analysis, Computer Methods in Applied Mechanics and Engineering 385 (2021) 114042.
}

\DIFaddend \bibitem{chen2001}
J.~S. Chen, C.~T. Wu, S.~Yoon, Y.~You, A stabilized conforming nodal integration for {{Galerkin}} mesh-free methods, International Journal for Numerical Methods in Engineering 50 (2001) 435--466.

\bibitem{chen2013a}
J.~S. Chen, M.~Hillman, M.~R{\"u}ter, An arbitrary order variationally consistent integration for {{Galerkin}} meshfree methods, International Journal for Numerical Methods in Engineering 95 (2013) 387--418.

\bibitem{duan2012a}
Q.~Duan, X.~Li, H.~Zhang, T.~Belytschko, Second-order accurate derivatives and integration schemes for meshfree methods, International Journal for Numerical Methods in Engineering 92 (2012) 399--424.

\bibitem{wang2019a}
D.~Wang, J.~Wu, An inherently consistent reproducing kernel gradient smoothing framework toward efficient {{Galerkin}} meshfree formulation with explicit quadrature, Computer Methods in Applied Mechanics and Engineering 349 (2019) 628--672.

\bibitem{wang2023}
J.~Wang, X.~Ren, A consistent projection integration for {{Galerkin}} meshfree methods, Computer Methods in Applied Mechanics and Engineering 414 (2023) 116143.

\bibitem{wu2022a}
J.~Wu, X.~Wu, Y.~Zhao, D.~Wang, A consistent and efficient method for imposing meshfree essential boundary conditions via hellinger-reissner variational principle., Chinese Journal of Theoretical and Applied Mechanics 54 (2022) 3283--3296.

\bibitem{wu2023}
J.~Wu, X.~Wu, Y.~Zhao, D.~Wang, A rotation-free {{Hellinger-Reissner}} meshfree thin plate formulation naturally accommodating essential boundary conditions, Engineering Analysis with Boundary Elements 154 (2023) 122--140.

\bibitem{benzaken2021}
J.~Benzaken, J.~A. Evans, S.~F. McCormick, R.~Tamstorf, Nitsche's method for linear {{Kirchhoff}}--{{Love}} shells: {{Formulation}}, error analysis, and verification, Computer Methods in Applied Mechanics and Engineering 374 (2021) 113544.

\bibitem{dah-wei1985}
H.~{Dah-wei}, A method for establishing generalized variational principle, Applied Mathematics and Mechanics 6 (1985) 501--509.

\bibitem{du2022}
H.~Du, J.~Wu, D.~Wang, J.~Chen, A unified reproducing kernel gradient smoothing {{Galerkin}} meshfree approach to strain gradient elasticity, Computational Mechanics 70 (2022) 73--100\DIFdelbegin \DIFdel{.
}%DIFDELCMD < 

%DIFDELCMD < \bibitem{kiendl2009}
\DIFdel{J.~Kiendl, K.~U. Bletzinger, J.~Linhard, R.~W}%DIFDELCMD < {%%%
\DIFdel{\"u}%DIFDELCMD < }%%%
\DIFdel{chner, Isogeometric shell analysis with }%DIFDELCMD < {{%%%
\DIFdel{Kirchhoff}%DIFDELCMD < }}%%%
\DIFdel{--}%DIFDELCMD < {{%%%
\DIFdel{Love}%DIFDELCMD < }} %%%
\DIFdel{elements, Computer Methods in Applied Mechanics and Engineering 198 (2009) 3902--3914}\DIFdelend .

\bibitem{macneal1985}
R.~H. Macneal, R.~L. Harder, A proposed standard set of problems to test finite element accuracy, Finite Elements in Analysis and Design 1 (1985) 3--20.

\end{thebibliography}
\end{document}
