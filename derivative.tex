\section{Green's theorems for in-plane vector}\label{appderivative}
This Appendix discuss two kinds of Green's theorems used for the development of the method. For an arbitrary vector $v^\alpha$ and a scalar function $f$, with the Green's theorem for in-plane vector, the first Green's theorem is list as follow \cite{benzaken2021}:
\begin{equation}
\begin{split}
\int_\Omega f_{,\alpha} v^\alpha d\Omega &= \int_\Gamma f v^\alpha n_\alpha d\Gamma - \int_\Omega f (v_{,\alpha}^\alpha + \Gamma^\beta_{\beta\alpha}v^\alpha) d\Omega \\
&= \int_\Gamma f v^\alpha n_\alpha d\Gamma - \int_\Omega f v^\alpha\vert_\alpha d\Omega
\end{split}
\end{equation}
where $\Gamma^\gamma_{\alpha\beta} = \boldsymbol a_{\alpha,\beta}\cdot \boldsymbol a^\gamma$ denotes the Christoffel symbol of the second kind. $v^\alpha\vert_\alpha$ can be regarded as the in-plane covariant derivative of the vector $v^\alpha$:
\begin{equation}
v^\alpha\vert_\alpha= v^\alpha_{,\alpha} + \Gamma^\beta_{\beta\alpha} v^\alpha
\end{equation}

The second Green's theorem is established with a mixed form of second order derivative, let $A^{\alpha\beta}$ be an arbitrary symmetric second order tensor, the Green's theorem yields \cite{benzaken2021}:
\begin{equation}\small
\begin{split}
\int_\Omega f_{,\alpha}\vert_\beta A^{\alpha\beta} d\Omega &= 
\int_\Gamma f_{,\gamma} n^\gamma A^{\alpha\beta} n_\alpha n_\beta d\Gamma 
- \int_\Gamma f(A^{\alpha\beta}s_\alpha n_\beta)_{,\gamma} s^\gamma d\Gamma
+ [[f A^{\alpha\beta} s_\alpha n_\beta]]_{\boldsymbol x \in C} \\
&- \int_\Gamma f(A^{\alpha\beta}_{,\beta}n_\alpha + \Gamma^\gamma_{\alpha\beta}A^{\alpha\beta}n_\gamma + \Gamma^\gamma_{\gamma\beta} A^{\alpha\beta} n_\alpha) d\Gamma \\
&+ \int_\Omega f \left (
\begin{split}
&\Gamma^\gamma_{\alpha\beta,\gamma}A^{\alpha\beta} + \Gamma^\gamma_{\alpha\beta} A^{\alpha\beta}_{,\gamma} + \Gamma^\eta_{\eta\gamma}\Gamma^\gamma_{\alpha\beta} A^{\alpha\beta} \\
+ &A^{\alpha\beta}_{,\alpha\beta} + \Gamma^\gamma_{\gamma\beta,\alpha}A^{\alpha\beta}+2\Gamma^\gamma_{\gamma\alpha}A^{\alpha\beta}_{,\beta} + \Gamma^{\gamma}_{\gamma\alpha}\Gamma^\eta_{\eta\beta} A^{\alpha\beta}
\end{split}
\right ) d\Omega \\
&=\int_\Gamma f_{,\gamma} n^\gamma A^{\alpha\beta} n_\alpha n_\beta d\Gamma 
- \int_\Gamma f(A^{\alpha\beta}s_\alpha n_\beta)_{,\gamma} s^\gamma d\Gamma
+ [[f A^{\alpha\beta} s_\alpha n_\beta]]_{\boldsymbol x \in C} \\
&-\int_\Gamma f A^{\alpha\beta}\vert_\beta n_\alpha d\Gamma
+ \int_\Omega f A^{\alpha\beta}\vert_{\alpha\beta} d\Omega
\end{split}
\end{equation}
with
\begin{equation}
A^{\alpha\beta}\vert_\beta = A^{\alpha\beta}_{,\beta} + \Gamma^\alpha_{\beta\gamma}A^{\beta\gamma} + \Gamma^\gamma_{\gamma\beta} A^{\alpha\beta}
\end{equation}
\begin{equation}
\begin{split}
A^{\alpha\beta}\vert_{\alpha\beta} = &\Gamma^\gamma_{\alpha\beta,\gamma}A^{\alpha\beta} + \Gamma^\gamma_{\alpha\beta} A^{\alpha\beta}_{,\gamma} + \Gamma^\eta_{\eta\gamma}\Gamma^\gamma_{\alpha\beta} A^{\alpha\beta} \\
+ &A^{\alpha\beta}_{,\alpha\beta} + \Gamma^\gamma_{\gamma\beta,\alpha}A^{\alpha\beta}+2\Gamma^\gamma_{\gamma\alpha}A^{\alpha\beta}_{,\beta} + \Gamma^{\gamma}_{\gamma\alpha}\Gamma^\eta_{\eta\beta} A^{\alpha\beta}
\end{split}
\end{equation}

For the sake of brevity, the notion of covariant derivative is extended to scalar function as:
\begin{equation}
f_{\vert\alpha} = f_{,\alpha} + \Gamma^\beta_{\beta\alpha} f \\
\end{equation}
\begin{equation}
f_{\vert\beta} n_\alpha = f_{,\beta}n_\alpha + \Gamma^\gamma_{\alpha\beta} f n_\gamma + \Gamma^\gamma_{\gamma_\beta} f n_\alpha
\end{equation}
\begin{equation}
\begin{split}
f_{\vert\alpha\beta} &=
\Gamma^\gamma_{\alpha\beta,\gamma}f + \Gamma^\gamma_{\alpha\beta} f_{,\gamma} + \Gamma^\eta_{\eta\gamma}\Gamma^\gamma_{\alpha\beta} f \\
&+ f_{,\alpha\beta} + \Gamma^\gamma_{\gamma\beta,\alpha}f+2\Gamma^\gamma_{\gamma\alpha}f_{,\beta} + \Gamma^{\gamma}_{\gamma\alpha}\Gamma^\eta_{\eta\beta} f
\end{split}
\end{align}
