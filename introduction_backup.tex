\section{Introduction}\label{introduction}
Thin shell structure follows the Kirchhoff hypothesis that neglects the shear deformation \cite{donnell1976}, which requires the approximation should have at least $C^1$ continuity in Galerkin formulations.
The traditional finite element methods usually only has $C^0$ continuous shape functions, and it more prefers Mindlin thick shear theory, hybrid and mixed models in simulation of shell structure \cite{hughes2000}.
In last three decades, the meshfree methods \cite{belytschko1994, liu1995, chen2017a} equipped high order smoothed shape functions have attracted significant research attention, while the meshfree shape functions are established based upon a set of scattered nodes and the high order continuity of shape functions is easily fulfilled even with low order basis function.
For thin shell analysis, this high order meshfree approximations can also alleviate the membrane locking caused by the mismatched approximation order of membrane strain and bending strain \cite{krysl1996}.
Moreover, in general, the nodal-based meshfree approximations can release the burden of mesh distortion and have the flexibility of local refinement.
Due to these advantages, a wide variety meshfree methods are proposed and have been applied to many scientific or engineering fields. 
However, the high order smoothed meshfree shape functions accompany with the enlarged and overlapping supports, which may also leads to many issues for shape functions. One is the loss of Kronecker delta property \cite{fernandez-mendez2004}, which leads to that the essential boundary conditions cannot be enforced directly like finite element methods.  
Another issue is that the variational consistency or said integration constraint cannot be satisfied, which is caused by the misalignment between numerical integration domains and supports of shape functions, and the shape functions exhibit a piecewise rational nature in each integration domains.
Variational consistency is of importance to the solution accuracy in Galerkin formulations \cite{wu2021}.

To directly enforce the essential boundary for Galerkin meshfree methods, several approaches have been proposed for the recovery of shape functions' Kronecker property. For examples, interpolation element-free method \cite{liu2019a}, mixed transformation method \cite{chen2000a}, boundary singular kernel method \cite{chen2000a} etc.
However, these methods are not based on a variational setting, and cannot guarantee the variational consistency, enforcing accuracy may be worse on where there is no meshfree node. In contrast, enforcing the essential boundary conditions by a variational approach are more preferred for Galerkin meshfree methods. 
Belytschko et al.  \cite{belytschko1994, krysl1996} firstly introduced the variational consistent Lagrange multiplier method to Galerkin meshfree method, in which the extra degrees of freedom should be employed for discretion of Lagrange multiplier. And this method has been extended to geometrically nonlinear thin shells by Ivannikov et al. \cite{ivannikov2014a}. 
To eliminate the extra degrees of freedom, Lu et al. \cite{lu1994} represented the Lagrange multiplier by corresponding tractions and proposed the modified variational essential boundary enforcement method. However, the coercivity of this approach is not always ensured and potentially reduces the accuracy. 
Zhu and Atluri \cite{zhu1998} pioneered the penalty method for meshfree method, making it straightforward approach for enforcing essential boundary conditions via Galerkin weak form. However, penalty method suffers from a lack of variational consistency, and requires the experimental artificial parameters, whose optimal value is hard to be determined.
Fernández-Méndez and Huerta \cite{fernandez-mendez2004} used the Nitsche's method in meshfree formulation for imposing essential boundary conditions. This method can be viewed as a hybrid of modified variational method and penalty method, since its consistent term that ensure variational consistency generated by modified variational method, and the penalty method is employed as stabilized term to recovery the coercivity. Skatulla and Sansour \cite{skatulla2008} further extended Nitsche's method for thin shell analysis and proposed an iteration algorithm to determine artificial parameters at each integration points.

To address the issue of numerical integration, a serial of consistent integration scheme has been developed for Galerkin meshfree methods. For instance, stabilized conforming nodal integration \cite{chen2001}, variational consistent integration \cite{chen2013}, quadratic consistent integration \cite{duan2012}, reproducing kernel gradient smoothing integration \cite{wang2019d}, consistent projection integration \cite{wang2023} etc.
The most consistent integration scheme is established by assumed strain approach, while the costly higher order derivatives of traditional meshfree shape functions are replaced by smoothed gradient, and show a high efficiency. 
Moreover, in order to achieve the global variational consistency, a consistent essential boundary condition enforcement should cooperate with the consistent integration scheme. 
The pair of consistent integration scheme and Nitsche's method for the treatment of essential boundary conditions shows a good performance, since it no needs the extra degrees of freedom and can fulfilled the coercivity. However,in Nitsche's method, the artificial parameters still exist in stabilized term and the costly higher order derivatives should be recalled, especially for thin plate and thin shell problems \cite{benzaken2021}.
Recently, Wu et al \cite{wu2022b, wu2023} proposed a efficient and stabilized essential boundary condition enforcement based upon the Hellinger-Reissner (HR) variational principle, where the reproducing kernel gradient smoothing integration is recast by a mixed formulation in Hellinger-Reissner weak form. The terms for enforcing essential boundary conditions is mostly identical with Nitsche's method, both have consistent term and stabilized term. Nevertheless, the stabilized term of this method naturally exist in Hellinger-Reissner weak form and no longer needs the artificial parameters, even for essential boundary enforcement, total of the higher order derivatives are represented by smoothed gradients and their derivatives.

In this study, an efficient and stabilized variational consistent meshfree method with naturally enforcing the essential boundary conditions is developed for thin shell structure. Follow the ideas of Hellinger-Reissner principle base consistent meshfree method, the Hu-Washizu variational principle of complementary energy \cite{dah-wei1985} with variables of displacement, strains and stresses is employed, where the displacement is approximated by conventional meshfree shape functions, and the strains and stresses are expressed by the smoothed gradients or covariant smoothed gradients with covariant bases. It should be noted that the smoothed gradients inherently embed the first two order integration constraints, however, due to the non-polynomial property of stresses, the fulfillment of these integration constraint only can get a quasi-satisfaction of variational consistency. All of the essential boundary conditions about displacements and rotations are considered in Hu-Washizu weak form, and present a Nitsche-like formalism but without any artificial parameters. Comparing with Nitsche's method, the costly higher order derivatives are replaced by conventional reproducing smoothed gradients and its direct derivatives. Taking the advantages of reproducing kernel gradient smoothing framework, the smoothed gradients shows a better performance on efficiency than conventional derivatives of shape functions, which improves the computational efficiency of meshfree formulation.

The remainder of this paper is organized as follows.
Section \ref{Kinematics} briefly describes the kinematics of thin shell structure and the corresponding Hu-Washizu principle weak form.
Subsequently, the mixed formulation regarding the displacements, strains and stresses in accordance with Hu-Washizu weak form is presented in Section \ref{mixed}.
Section \ref{boundary} derives the discrete equilibrium equations with the naturally accommodation of essential, and compares them with those of Nitsche's method.
The efficacy of the proposed Hu-Washizu meshfree thin shell formulation is validated by numerical results in Section \ref{examples}.
Concluding remarks are finally drawn in Section \ref{conclusion}.
