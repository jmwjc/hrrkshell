\section{Mixed meshfree formulation for modified Hellinger-Reissner weak form}\label{mixed}
\subsection{Reproducing kernel approximation for displacement}
In this study, the displacement is approximated by traditional reproducing kernel approximation. As shown in Fig, the mid-surface of the shell $\Omega$ is discretized by a set of meshfree nodes $\{\boldsymbol \xi_I\}_{I=1}^{n_p}$ in parametric configuration, where $n_p$ is the total number of meshfree nodes. The approximated displacement namely $\boldsymbol v^h$ can be expressed by:
\begin{equation}\label{approxv}
\boldsymbol v(\boldsymbol \xi) = \sum_{I=1}^{n_p} \Psi_I(\boldsymbol \xi) \boldsymbol d_I
\end{equation}
in which $\Psi_I$ and $\boldsymbol d_I$ is the shape function and nodal coefficient tensor related by node $\boldsymbol \xi_I$. According to reproducing kernel approximation \cite{liu1995}, the shape function takes the following form:
\begin{equation}
\Psi_I(\boldsymbol \xi) = \boldsymbol p^T(\boldsymbol \xi) \boldsymbol c(\boldsymbol \xi) \phi(\boldsymbol \xi_I - \boldsymbol \xi)
\end{equation}
where $\boldsymbol p$ is the basis function vector, and in this study, the following quadratic basis function is considered:
\begin{equation}
        \boldsymbol p = \{1,\;\xi^1,\;\xi^2,\;(\xi^1)^2,\xi^1\xi^2,(\xi^2)^2\}^T
\end{equation}

The kernel function denoted by$\phi$ controls the support and smoothness of meshfree shape functions. The quantic B-spline function with square support is used herein as the kernel function:
\begin{equation}
\phi(\boldsymbol \xi_I - \boldsymbol \xi) = \phi(\hat s_1)\phi(\hat s_2), \quad \hat s_\alpha = \frac{\vert \xi^\alpha_I - \xi^\alpha\vert}{s_{\alpha I}}
\end{equation}
with
\begin{equation}
\phi(\hat s_\alpha) = \frac{1}{5!}\begin{cases}
(3-3\hat s_\alpha)^5 - 6(2-3\hat s_\alpha)^5 + 15(1-3\hat s_\alpha)^5 & \hat s_\alpha \le \frac{1}{3} \\
(3-3\hat s_\alpha)^5 - 6(2-3\hat s_\alpha)^5 & \frac{1}{3}<\hat s_\alpha \le \frac{2}{3} \\
(3-3\hat s_\alpha)^5 & \frac{2}{3}<\hat s_\alpha \le 1 \\
0 & \hat s_\alpha >1
\end{cases}
\end{equation}
and $\hat s_{\alpha I}$ means the characterized size of support for meshfree shape function $\Psi_I$.

The unknown vector $\boldsymbol c$ in shape function are determined by the fulfillment of the so-call consistency condition:
\begin{equation}
\sum_{I=1}^{n_p} \Psi_I(\boldsymbol \xi)\boldsymbol p(\boldsymbol \xi_I) = \boldsymbol p(\boldsymbol \xi)
\end{equation}
or equivalently
\begin{equation}\label{cc}
\sum_{I=1}^{n_p} \Psi_I(\boldsymbol \xi)\boldsymbol p(\boldsymbol \xi_I-\boldsymbol \xi) = \boldsymbol p(\boldsymbol 0)
\end{equation}
Substituting Eq. (\ref{approxv}) into (\ref{cc}), yields:
\begin{equation}\label{A}
\boldsymbol A(\boldsymbol \xi) \boldsymbol c(\boldsymbol \xi) = \boldsymbol p(\boldsymbol 0)\quad \Rightarrow \quad
\boldsymbol c(\boldsymbol \xi) = \boldsymbol A^{-1}(\boldsymbol \xi)\boldsymbol p(\boldsymbol 0)
\end{equation}
where $\boldsymbol A$ is the moment matrix:
\begin{equation}
\boldsymbol A(\boldsymbol \xi) = \sum_{I=1}^{n_p}\phi(\boldsymbol \xi_I - \boldsymbol \xi) \boldsymbol p(\boldsymbol \xi_I-\boldsymbol \xi)\boldsymbol p^T(\boldsymbol \xi_I - \boldsymbol \xi)
\end{equation}
Taking Eq. (\ref{A}) back into Eq. (\ref{approxv}), the expression of meshfree shape function can be given by:
\begin{equation}
\Psi_I(\boldsymbol \xi) = \boldsymbol p^T(\boldsymbol \xi_I - \boldsymbol \xi)\boldsymbol A^{-1}(\boldsymbol \xi) \boldsymbol p(\boldsymbol 0) \phi(\boldsymbol \xi_I-\boldsymbol \xi)
\end{equation}
\subsection{Reproducing kernel gradient smoothing approximation for effective stress and strain}
In Galerkin meshfree formulation, the mid-plane of thin shell $\Omega$ is split by a set of integration cells $\Omega_C$'s, $\cup_{C=1}^{n_e}\Omega_C\approx \Omega$. With the inspiration of reproducing kernel smoothing framework, the Cartesian and covariant derivatives of displacement, $\boldsymbol v_{,\alpha}$ and $-\boldsymbol v_{,\alpha}\vert_\beta$, in strains $\varepsilon_{\alpha\beta}$, $\kappa_{\alpha\beta}$ are approximated by $(p-1)$-th order polynomials in each integration cells. In integration cell $\Omega_C$, the approximated derivatives and strains denoted by $\boldsymbol v^h_{,\alpha}$, $\varepsilon^h_{\alpha\beta}$ and $-\boldsymbol v^h_{,\alpha}\vert_\beta$, $\kappa^h_{\alpha\beta}$ can be expressed by:
\begin{equation}\label{approxsn1}
    \boldsymbol v^h_{,\alpha}(\boldsymbol \xi) = \boldsymbol q^T(\boldsymbol \xi) \boldsymbol d_{\alpha}^\varepsilon, \quad
    \varepsilon^h_{\alpha\beta}(\boldsymbol \xi) = \boldsymbol q^T(\boldsymbol \xi) \frac{1}{2}(\boldsymbol a_\alpha \cdot \boldsymbol d_{\beta}^\varepsilon + \boldsymbol a_\beta \cdot \boldsymbol d_{\alpha}^\varepsilon)
\end{equation}
\begin{equation}\label{approxsn2}
    -\boldsymbol v^h_{,\alpha}\vert_\beta(\boldsymbol \xi) = \boldsymbol q^T(\boldsymbol \xi) \boldsymbol d_{\alpha\beta}^\kappa , \quad
    \kappa^h_{\alpha\beta}(\boldsymbol \xi) = \boldsymbol q^T(\boldsymbol \xi) \boldsymbol a_3 \cdot \boldsymbol d_{\alpha\beta}^\kappa
\end{equation}
where $\boldsymbol q$ is the $(p-1)$th order polynomial vector and has the following form:
\begin{equation}
\boldsymbol q = \{ 1,\; \xi^1,\; \xi^2,\; \dots, (\xi^2)^{p-1}\}^T
\end{equation}
and the $\boldsymbol d^\varepsilon_{\alpha}$, $\boldsymbol d^\kappa_{\alpha\beta}$ are the corresponding coefficient vector tensors. For the conciseness, the mixed usage of tensor and vector is introduced in this study, for example, the component of coefficient tensor vector $\boldsymbol d^\varepsilon_{\alpha I}$, $\boldsymbol d^\varepsilon_\alpha = \{\boldsymbol d^\varepsilon_{\alpha I}\}$, is a three dimensional tensor, $\dim \boldsymbol d^\varepsilon_{\alpha I} = \dim \boldsymbol v$.

In order to meet the integration constraint of thin shell problem, the approximated stresses $N^{\alpha\beta h}$, $M^{\alpha\beta h}$ are assumed to be a similar form with strains, yields:
\begin{equation}\label{approxse1}
N^{\alpha\beta h}(\boldsymbol \xi) = \boldsymbol q^T(\boldsymbol \xi) \boldsymbol a^\alpha \cdot \boldsymbol d_{\beta}^N,\quad
\boldsymbol a_\alpha N^{\alpha\beta h}(\boldsymbol \xi) = \boldsymbol q^T(\boldsymbol \xi) \boldsymbol d^N_\beta
\end{equation}
\begin{equation}\label{approxse2}
    M^{\alpha\beta h}(\boldsymbol \xi) = \boldsymbol q^T(\boldsymbol \xi) \boldsymbol a_3 \cdot \boldsymbol d_{\alpha\beta}^M,\quad
    \boldsymbol a_3 M^{\alpha\beta h}(\boldsymbol \xi) = \boldsymbol q^T(\boldsymbol \xi) \boldsymbol d_{\alpha\beta}^M
\end{equation}
substituting the approximations of Eqs. (\ref{approxv}), (\ref{approxsn1}), (\ref{approxsn2}), (\ref{approxse1}), (\ref{approxse2}) into Eqs. (\ref{w3}), (\ref{w4}) can express $\boldsymbol d^\varepsilon_\beta$ and $\boldsymbol d^\kappa_{\alpha\beta}$ by $\boldsymbol d$ as:
\begin{equation}\label{depsilon}
\boldsymbol d^\varepsilon_\beta = \boldsymbol G^{-1} \left (\sum_{I=1}^{n_p}(\tilde{\boldsymbol g}_{\beta I} - \bar{\boldsymbol g}_{\beta I}) \boldsymbol d_I + \hat{\boldsymbol g}_\beta \right )
\end{equation}
\begin{equation}\label{dkappa}
\boldsymbol d^\kappa_{\alpha\beta} = \boldsymbol G^{-1} \left (\sum_{I=1}^{n_p}(\tilde{\boldsymbol g}_{\alpha\beta I} - \bar{\boldsymbol g}_{\alpha\beta I})\boldsymbol d_I + \hat{\boldsymbol g}_{\alpha\beta} \right )
\end{equation}
with
\begin{equation}
\boldsymbol G = \int_{\Omega_C} \boldsymbol q^T \boldsymbol q d\Omega
\end{equation}
\begin{subequations}
\begin{align}
\tilde{\boldsymbol g}_{\beta I} &= \int_{\Gamma_C} \Psi_I \boldsymbol q n_\beta d\Gamma
- \int_{\Omega_C} \Psi_I \boldsymbol q^*\vert_\beta d\Omega \\
\bar{\boldsymbol g}_{\beta I} &= \int_{\Gamma_C\cap\Gamma_v} \Psi_I \boldsymbol q n_\beta d\Gamma \\
\hat{\boldsymbol g}_{\beta} &= \int_{\Gamma_C\cap\Gamma_v} \boldsymbol q n_\beta \bar{\boldsymbol v} d\Gamma 
\end{align}
\end{subequations}
\begin{subequations}
\begin{align}
\small
\begin{split}
\tilde{\boldsymbol g}_{\alpha\beta I} &= \int_{\Gamma_C} \Psi_{I,\gamma}n^\gamma \boldsymbol q n_\alpha n_\beta d\Gamma 
- \int_{\Gamma_C} \Psi_I(\boldsymbol q^{**}\vert_\beta n_\alpha + (\boldsymbol q s_\alpha n_\beta)_{,\gamma}s^\gamma) d\Gamma \\
&+ [[\Psi_I \boldsymbol q s_\alpha n_\beta]]_{\boldsymbol x\in C_C}
- \int_{\Omega_C} \Psi \boldsymbol q^{**}_{,\alpha}\vert_\beta d\Omega \\
\end{split} \\
\small
\begin{split}
\bar{\boldsymbol g}_{\alpha\beta I} &= \int_{\Gamma_C\cap\Gamma_\theta} \Psi_{I,\gamma}n^\gamma \boldsymbol q n_\alpha n_\beta d\Gamma 
- \int_{\Gamma_C\cap\Gamma_v} \Psi_I(\boldsymbol q^{**}\vert_\beta n_\alpha + (\boldsymbol q s_\alpha n_\beta)_{,\gamma}s^\gamma) d\Gamma \\
&+ [[\Psi_I \boldsymbol q s_\alpha n_\beta]]_{\boldsymbol x\in C_C\cap C_v}
\end{split} \\
\small
\begin{split}
\hat{\boldsymbol g}_{\alpha\beta} &= \int_{\Gamma_C\cap\Gamma_\theta} \boldsymbol q n_\alpha n_\beta \boldsymbol a_3 \bar{\theta}_{\boldsymbol n} d\Gamma 
- \int_{\Gamma_C\cap\Gamma_v}(\boldsymbol q^{**}\vert_\beta n_\alpha + (\boldsymbol q s_\alpha n_\beta)_{,\gamma}s^\gamma)\bar{\boldsymbol v} d\Gamma \\
&+ [[\boldsymbol q s_\alpha n_\beta \bar{\boldsymbol v}]]_{\boldsymbol x\in C_C\cap C_v}
\end{split}
\end{align}
\end{subequations}
plugging Eqs. (\ref{depsilon}) and (\ref{dkappa}) back into Eqs. (\ref{approxsn1}) and (\ref{approxsn2}) respectively gives the final expression of $\boldsymbol v^h_{,\alpha}$, $\varepsilon^h_{\alpha\beta}$ and $-\boldsymbol v^h_{,\alpha\beta}$, $\boldsymbol \kappa^h_{\alpha\beta}$ as:
\begin{subequations}
\begin{equation}
\boldsymbol v^h_{,\alpha} = \sum_{I=1}^{n_p}(
\tilde \Psi_{I,\alpha} - \bar \Psi_{I,\alpha}) \boldsymbol d_I +
\boldsymbol q^T \boldsymbol G^{-1}\hat{\boldsymbol g}_{\alpha}
\end{equation}
\begin{equation}\label{epsilonh}
\begin{split}
\varepsilon^h_{\alpha\beta} &= 
\sum_{I=1}^{n_p} \frac{1}{2}(\boldsymbol a_\alpha \tilde \Psi_{I,\beta} + \boldsymbol a_\beta \tilde \Psi_{I,\alpha}) \cdot \boldsymbol d_I 
- \sum_{I=1}^{n_p} \frac{1}{2}(\boldsymbol a_\alpha \bar \Psi_{I,\beta} + \boldsymbol a_\beta \bar \Psi_{I,\alpha}) \cdot \boldsymbol d_I \\
&+ \boldsymbol q^T \boldsymbol G^{-1} \frac{1}{2}(\boldsymbol a_\alpha \cdot \hat{\boldsymbol g}_{\beta} + \boldsymbol a_\beta \cdot \hat{\boldsymbol g}_{\alpha}) \\
&= \tilde \varepsilon^h_{\alpha\beta} - \bar \varepsilon^h_{\alpha\beta} + \hat \varepsilon^h_{\alpha\beta}
\end{split}
\end{equation}
\end{subequations}
\begin{subequations}
\begin{equation}
-\boldsymbol v^h_{,\alpha}\vert_\beta = \sum_{I=1}^{n_p} (
\tilde \Psi_{I,\alpha\beta} -
\bar \Psi_{I,\alpha\beta} ) \boldsymbol d_I +
\boldsymbol q^T \boldsymbol G^{-1}\hat{\boldsymbol g}_{\alpha\beta}
\end{equation}
\begin{equation}\label{kappah}
\begin{split}
\kappa^h_{\alpha\beta} &= \sum_{I=1}^{n_p} \tilde \Psi_{I,\alpha\beta} \boldsymbol a_3 \cdot \boldsymbol d_I
- \sum_{I=1}^{n_p} \bar \Psi_{I,\alpha\beta} \boldsymbol a_3 \cdot \boldsymbol d_I +
\boldsymbol q^T \boldsymbol G^{-1}\boldsymbol a_3 \cdot \hat{\boldsymbol g}_{\alpha\beta} \\
&= \tilde \kappa^h_{\alpha\beta} - \bar \kappa^h_{\alpha\beta} + \hat \kappa^h_{\alpha\beta}
\end{split}
\end{equation}
\end{subequations}
with
\begin{equation}
\tilde{\Psi}_{I,\alpha}(\boldsymbol \xi) = \boldsymbol q^T(\boldsymbol \xi) \boldsymbol G^{-1} \tilde{\boldsymbol g}_{\alpha I}, \quad
\bar{\Psi}_{I,\alpha}(\boldsymbol \xi) = \boldsymbol q^T(\boldsymbol \xi) \boldsymbol G^{-1} \tilde{\boldsymbol g}_{\alpha I}
\end{equation}
\begin{equation}
\left \{
\begin{split}
\tilde \varepsilon^h_{\alpha\beta} &= \sum_{I=1}^{n_p} \frac{1}{2}(\boldsymbol a_\alpha \tilde \Psi_{I,\beta} + \boldsymbol a_\beta \tilde \Psi_{I,\alpha}) \cdot \boldsymbol d_I \\
\bar \varepsilon^h_{\alpha\beta} &= \sum_{I=1}^{n_p} \frac{1}{2}(\boldsymbol a_\alpha \bar \Psi_{I,\beta} + \boldsymbol a_\beta \bar \Psi_{I,\alpha}) \cdot \boldsymbol d_I \\
\hat \varepsilon^h_{\alpha\beta} &= \boldsymbol q^T \boldsymbol G^{-1} \frac{1}{2}(\boldsymbol a_\alpha\cdot\hat{\boldsymbol g}_\beta + \boldsymbol a_\beta \cdot \hat{\boldsymbol g}_\alpha)
\end{split}
\right .
\end{equation}
\begin{equation}
\tilde{\Psi}_{I,\alpha\beta}(\boldsymbol \xi) = \boldsymbol q^T(\boldsymbol \xi) \boldsymbol G^{-1} \tilde{\boldsymbol g}_{\alpha\beta I}, \quad
\bar{\Psi}_{I,\alpha\beta}(\boldsymbol \xi) = \boldsymbol q^T(\boldsymbol \xi) \boldsymbol G^{-1} \tilde{\boldsymbol g}_{\alpha\beta I}
\end{equation}
\begin{equation}
\left \{
\begin{split}
\tilde \kappa^h_{\alpha\beta} &= \sum_{I=1}^{n_p} \tilde \Psi_{I,\alpha\beta}\boldsymbol a_3 \cdot \boldsymbol d_I \\
\bar \kappa^h_{\alpha\beta} &= \sum_{I=1}^{n_p} \bar \Psi_{I,\alpha\beta}\boldsymbol a_3 \cdot \boldsymbol d_I \\
\hat \kappa^h_{\alpha\beta} &= \boldsymbol q^T \boldsymbol G^{-1} \boldsymbol a_3 \cdot \hat{\boldsymbol g}_{\alpha\beta}
\end{split}
\right .
\end{equation}

Furthermore, taking Eqs. (\ref{approxsn1}) and (\ref{approxsn2}) into Eqs.(\ref{w1}) and (\ref{w2}) can obtain the approximated effective stresses $N^{\alpha\beta h}$, $M^{\alpha\beta h}$ and their coefficients $\boldsymbol d^N_\beta$, $\boldsymbol d^M_{\alpha\beta}$ as:
\begin{equation}
\begin{split}
&\frac{1}{2}(\delta \boldsymbol d^\varepsilon_\alpha \cdot \boldsymbol a_\beta + \delta \boldsymbol d^\varepsilon_\beta\ \cdot \boldsymbol a_\alpha)
h C^{\alpha\beta\gamma\eta}
\frac{1}{2}(\boldsymbol a_\gamma \cdot \boldsymbol d^\varepsilon_\eta + \boldsymbol a_\gamma \cdot \boldsymbol d^\varepsilon_\eta)
\boldsymbol G \\ = &
\frac{1}{2}(\delta \boldsymbol d^\varepsilon_\alpha \cdot \boldsymbol d^N_\beta + \delta \boldsymbol d^\varepsilon_\beta \cdot \boldsymbol d^N_\alpha) \boldsymbol G \\
\Rightarrow \; \boldsymbol d_N^\beta = &\boldsymbol a_\beta hC^{\alpha\beta\gamma\eta} \frac{1}{2}(\boldsymbol a_\gamma \cdot \boldsymbol d^\varepsilon_\eta + \boldsymbol a_\eta \cdot \boldsymbol d^\varepsilon_\gamma)
\end{split} 
\end{equation}
\begin{equation}
\begin{split}
& \delta \boldsymbol d^\kappa_{\alpha\beta}\cdot \boldsymbol a_3 \frac{h^3}{12} C^{\alpha\beta\gamma\eta} \boldsymbol a_3 \cdot \boldsymbol d^\kappa_{\gamma\eta} \boldsymbol G = \delta \boldsymbol d^\kappa_{\alpha\beta} \cdot \boldsymbol d^M_{\alpha\beta} \boldsymbol G \\
\Rightarrow \; &\boldsymbol d_M^{\alpha\beta} = \boldsymbol a_3 \frac{h^3}{12}C^{\alpha\beta\gamma\eta}\boldsymbol a_3 \cdot \boldsymbol d^\kappa_{\gamma\eta}
\end{split}
\end{equation}
\begin{equation}\label{Nh}
N^{\alpha\beta h} &= hC^{\alpha\beta\gamma\eta}(\tilde \varepsilon^h_{\gamma\eta} - \bar \varepsilon^h_{\gamma\eta} + \hat \varepsilon^h_{\gamma\eta}) = \tilde N^{\alpha\beta h} - \bar N^{\alpha\beta h} + \hat N^{\alpha\beta h}
\end{equation}
\begin{equation}\label{Mh}
M^{\alpha\beta h} &= \frac{h^3}{12}C^{\alpha\beta\gamma\eta}(\tilde \kappa^h_{\gamma\eta} - \bar \kappa^h_{\gamma\eta} + \hat \kappa^h_{\gamma\eta}) = \tilde M^{\alpha\beta h} - \bar M^{\alpha\beta h} + \hat M^{\alpha\beta h}
\end{equation}
with
\begin{equation}
\tilde N^{\alpha\beta h} = hC^{\alpha\beta\gamma\eta}\tilde \varepsilon^h_{\gamma\eta} ,\quad
\bar N^{\alpha\beta h} = hC^{\alpha\beta\gamma\eta}\bar \varepsilon^h_{\gamma\eta} ,\quad
\hat N^{\alpha\beta h} = hC^{\alpha\beta\gamma\eta}\hat \varepsilon^h_{\gamma\eta}
\end{equation}
\begin{equation}
\tilde M^{\alpha\beta h} = \frac{h^3}{12}C^{\alpha\beta\gamma\eta}\tilde \kappa^h_{\gamma\eta} ,\quad
\bar M^{\alpha\beta h} = \frac{h^3}{12}C^{\alpha\beta\gamma\eta}\bar \kappa^h_{\gamma\eta} ,\quad
\hat M^{\alpha\beta h} = \frac{h^3}{12}C^{\alpha\beta\gamma\eta}\hat \kappa^h_{\gamma\eta}
\end{equation}

It is noted that, referring to reproducing kernel gradient smoothing framework \cite{wang2019d}, $\tilde \Psi_{I,\alpha}$, $\tilde \Psi_{I,\alpha\beta}$ are actually the first and second order smoothed gradients in curvilinear coordinates. $\tilde{\boldsymbol g}_{\alpha I}$ and $\tilde{\boldsymbol g}_{\alpha \beta I}$ are the right hand side integration constraints for first and second order gradients, then this formulation can meet the variational consistency for the $p$-th order polynomials. It should be known that, in curved model, the variational consistency for non-polynomial functions, like trigonometric functions, should be required for the polynomial solution. Even with $p$-th order variational consistency, the proposed formulation can not exactly reproduce the solution spanned by basis functions, however the accuracy of reproducing kernel smoothed gradients is still better that traditonal meshfree formulation, this will be evidenced by numerical examples in further section.
