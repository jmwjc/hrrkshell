\section{Mixed meshfree formulation for modified Hellinger-Reissner weak form}
\subsection{Reproducing kernel approximation for displacement}
In this study, the displacement is approximated by traditional reproducing kernel approximation. As shown in Fig, 

\begin{equation}
\boldsymbol v(\boldsymbol \xi) = \sum_{I=1}^{n_p} \Psi_I(\boldsymbol \xi) \boldsymbol d_I
\end{equation}

\subsection{Reproducing kernel gradient smoothing approximation for effective stress and strain}
In Galerkin meshfree formulation, the mid-plane of thin shell $\Omega$ is split by a set of integration cells $\Omega_C$'s, $\cup_{C=1}^{n_e}\Omega_C\approx \Omega$. With the inspiration of reproducing kernel smoothing framework, the Cartesian and covariant derivatives of displacement, $\boldsymbol v_{,\alpha}$ and $-\boldsymbol v_{,\alpha}\vert_\beta$, in strains $\varepsilon_{\alpha\beta}$, $\kappa_{\alpha\beta}$ are approximated by $(p-1)$-th order polynomials in each integration cells. The approximated derivatives and strains denoted by $\boldsymbol v^h_{,\alpha}$, $\varepsilon_{\alpha\beta}$ and $-\boldsymbol v_{,\alpha}\vert_\beta$, $\kappa_{\alpha\beta}$ can be expressed by:

\begin{equation}
    \boldsymbol v_{,\alpha}(\boldsymbol \xi) = \boldsymbol q^T(\boldsymbol \xi) \boldsymbol d_{\alpha}^\varepsilon, \quad
    \varepsilon_{\alpha\beta}(\boldsymbol \xi) = \boldsymbol q^T(\boldsymbol \xi) \frac{1}{2}(\boldsymbol a_\alpha \cdot \boldsymbol d_{\beta}^\varepsilon + \boldsymbol a_\beta \cdot \boldsymbol d_{\alpha}^\varepsilon)
\end{equation}

\begin{equation}
    -\boldsymbol v_{,\alpha}\vert_\beta(\boldsymbol \xi) = \boldsymbol q^T(\boldsymbol \xi) \boldsymbol d_{\alpha\beta}^\kappa , \quad
    \kappa^{\alpha\beta}(\boldsymbol \xi) = - \boldsymbol q^T(\boldsymbol \xi) \boldsymbol a_3 \cdot \boldsymbol d_{\alpha\beta}^\kappa
\end{equation}

where $\boldsymbol q$ is the $(p-1)$th order polynomial vector and has the following form:

\begin{equation}
\boldsymbol q = \{ 1,\; \xi^1,\; \xi^2,\; \dots, (\xi^2)^{p-1}\}^T
\end{equation}

and the 
\begin{equation}
N^{\alpha\beta}(\boldsymbol x) = \boldsymbol p^T(\boldsymbol x) \boldsymbol a^\alpha \cdot \boldsymbol d_{\beta}^N,\quad
\boldsymbol a_\alpha N^{\alpha\beta} = \boldsymbol p^T(\boldsymbol x) \boldsymbol d^N_\beta
\end{equation}
\begin{equation}
    M^{\alpha\beta}(\boldsymbol x) = \boldsymbol p^T(\boldsymbol x) \boldsymbol a_3 \cdot \boldsymbol d_{\alpha\beta}^M,\quad
    \boldsymbol a_3 M^{\alpha\beta} = \boldsymbol p^T(\boldsymbol x) \boldsymbol d_{\alpha\beta}^M
\end{equation}
