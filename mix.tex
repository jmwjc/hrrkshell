\section{Mixed meshfree formulation for modified Hellinger-Reissner weak form}
\subsection{Reproducing kernel approximation for displacement}
In this study, the displacement is approximated by traditional reproducing kernel approximation. As shown in Fig, 

\begin{equation}
\boldsymbol v(\boldsymbol x) = \sum_{I=1}^{n_p} \Psi_I(\boldsymbol x) \boldsymbol d_I
\end{equation}

\begin{equation}
    \boldsymbol v_{,\alpha}(\boldsymbol x) = \boldsymbol p^T(\boldsymbol x) \boldsymbol d_{\alpha}^\varepsilon, \quad
    \varepsilon_{\alpha\beta}(\boldsymbol x) = \boldsymbol p^T(\boldsymbol x) \frac{1}{2}(\boldsymbol a_\alpha \cdot \boldsymbol d_{\beta}^\varepsilon + \boldsymbol a_\beta \cdot \boldsymbol d_{\alpha}^\varepsilon)
\end{equation}

\subsection{Reproducing kernel gradient smoothing approximation for effective stress and strain}
For the inspiration 
\begin{equation}
    -(\boldsymbol v_{,\alpha})\vert_\beta(\boldsymbol x) = \boldsymbol p^T(\boldsymbol x) \boldsymbol d_{\alpha\beta}^\kappa , \quad
    \kappa^{\alpha\beta}(\boldsymbol x) = - \boldsymbol p^T(\boldsymbol x) \boldsymbol a_3 \cdot \boldsymbol d_{\alpha\beta}^\kappa
\end{equation}
\begin{equation}
N^{\alpha\beta}(\boldsymbol x) = \boldsymbol p^T(\boldsymbol x) \boldsymbol a^\alpha \cdot \boldsymbol d_{\beta}^N,\quad
\boldsymbol a_\alpha N^{\alpha\beta} = \boldsymbol p^T(\boldsymbol x) \boldsymbol d^N_\beta
\end{equation}
\begin{equation}
    M^{\alpha\beta}(\boldsymbol x) = \boldsymbol p^T(\boldsymbol x) \boldsymbol a_3 \cdot \boldsymbol d_{\alpha\beta}^M,\quad
    \boldsymbol a_3 M^{\alpha\beta} = \boldsymbol p^T(\boldsymbol x) \boldsymbol d_{\alpha\beta}^M
\end{equation}
