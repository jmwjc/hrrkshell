
\begin{frontmatter}
    \title{A Hu-Washizu variational consistent meshfree thin shell formulation with naturally accommodating essential boundary conditions}
    \author[1]{Junchao Wu\corref{cor1}}
    \ead{jcwu@hqu.edu.cn}
    \author[2]{Yangtao Xu}
    \author[1]{Bin Xu}
    \author[3]{Syed Humayun Basha}

    \affiliation[1]{organization={Key Laboratory for Intelligent Infrastructure and Monitoring of Fujian Province,
                                  College of Civil Engineering,
                                  Huaqiao University},
                    city={Xiamen},
                    state={Fujian},
                    postcode={361021},
                    country={China}}

    \affiliation[2]{organization={College of Civil Engineering,
                                  Huaqiao University},
                    city={Xiamen},
                    state={Fujian},
                    postcode={361021},
                    country={China}}

    \affiliation[3]{organization={Key Laboratory for Structural Engineering and Disaster Prevention of Fujian Province,
                                  College of Civil Engineering,
                                  Huaqiao University},
                    city={Xiamen},
                    state={Fujian},
                    postcode={361021},
                    country={China}}

    \cortext[cor1]{Corresponding author}

\begin{abstract}
A Hu-Washizu principle based variational consistent meshfree formulation with naturally enforcement of essential boundary conditions is proposed for thin shell analysis. In this approach, a mixed formulation of displacements, strains and stresses within the framework of Hu-Washizu variational principle is employed, where the displacements are discretized by meshfree shape functions, the strains and stresses are expressed as the smoothed gradients and covariant smoothed gradients which meet the first two order integration constraint and have the quasi- varational consistency. 
The numerical examples demonstrate the efficacy and efficiency of proposed method, while the RKGSI performs a comparable result in energy error with interpolation by meshfree approximations.
\end{abstract}
\begin{keyword}
Meshfree \sep Thin shell \sep Hu-Washizu variational principle \sep Reproducing kernel gradient smoothing \sep Essential boundary condition
\end{keyword}
\end{frontmatter}
