
\begin{frontmatter}
    % \title{A Hu-Washizu variational consistent meshfree thin shell formulation naturally accommodating essential boundary conditions}
    % \title{Quasi-consistent efficient meshfree thin shell formulation to naturally accommodate essential boundary conditions}
    % \title{Quasi-consistent efficient meshfree thin shell formulation with penalty-free essential boundary condition enforcement}
    \title{Quasi-consistent efficient meshfree thin shell formulation with naturally stabilized enforced essential boundary conditions}
    \author[1]{Junchao Wu\corref{cor1}}
    \ead{jcwu@hqu.edu.cn}
    \author[1]{Yangtao Xu}
    \author[1]{Bin Xu}
    \author[1]{Syed Humayun Basha}

    \affiliation[1]{organization={Key Laboratory for Intelligent Infrastructure and Monitoring of Fujian Province,
                                  Key Laboratory for Structural Engineering and Disaster Prevention of Fujian Province,
                                  College of Civil Engineering,
                                  Huaqiao University},
                    city={Xiamen},
                    state={Fujian},
                    postcode={361021},
                    country={China}}

    \cortext[cor1]{Corresponding author}

\begin{abstract}
This research proposed an efficient and quasi-consistent meshfree thin shell formulation with naturally stabilized enforcement of essential boundary conditions. Within the framework of the Hu-Washizu variational principle, a mixed formulation of displacements, strains and stresses is employed in this approach, where the displacements are discretized using meshfree shape functions, and the strains and stresses are expressed using smoothed gradients and covariant bases. The smoothed gradients satisfy the first second-order integration constraint and observed variational consistency for polynomial strains and stresses. Owing to Hu-Washizu variational principle, the essential boundary conditions automatically arise in its weak form. As a result, the suggested technique's enforcement of essential boundary conditions resembles that of the traditional Nitsche's method. Contrary to Nitsche's method, the costly higher order derivatives of conventional meshfree shape functions are replaced by the smoothed gradients with fast computation, which improve the efficiency. Meanwhile, the proposed formulation features a naturally stabilized term without adding any artificial stabilization factors, which eliminates the application of penalty method as a stabilization. Further, the efficacy of the proposed Hu-Washizu meshfree thin shell formulation is illustrated by a set of classical standard thin shell problems.
\end{abstract}
\begin{keyword}
Meshfree \sep Thin shell \sep Hu-Washizu variational principle \sep Reproducing kernel gradient smoothing \sep Essential boundary condition
\end{keyword}
\end{frontmatter}
