\section{Conclusion}\label{conclusion}
In this study, an efficient and quasi-consistent meshfree thin shell formulation was presented to naturally enforce the essential boundary conditions.  Mixed formulation with the Hu-Washizu principle weak form is adopted, where the traditional meshfree shape functions discretized the displacement, and the strains and stresses were expressed by the reproducing kernel smoothed gradients and the covariant bases, respectively. The smoothed gradient naturally embedded the first second-order integration constraints and has a quasi variational consistency for the curved models in each integration cell. Owing to the Hu-Washizu variational principle, the essential boundary condition enforcement has a similar form with the conventional Nitsche’s method; both have consistent and stabilized terms. The costly high order derivatives in the Nitsche’s consistent term have been replaced by the smoothed gradients, which improved the computational speed due to the reproducing kernel gradient smoothing framework. Furthermore, the stabilized term naturally existed in the Hu-Washizu weak form, and the artificial parameter needed in Nitsche’s stabilized term has vanished, which can automatically maintain the coercivity for the stiffness matrix. Based on general reproducing kernel gradient smoothing framework, the proposed methodology can be trivially extended to high order basis meshfree formulation. The numerical results demonstrated that the proposed Hu-Washizu quasi-consistent meshfree thin shell formulation showed excellent accuracy, efficiency, and stability.

